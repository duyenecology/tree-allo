% Options for packages loaded elsewhere
\PassOptionsToPackage{unicode}{hyperref}
\PassOptionsToPackage{hyphens}{url}
\PassOptionsToPackage{dvipsnames,svgnames,x11names}{xcolor}
%
\documentclass[
  12pt,
  letterpaper,
  DIV=11,
  numbers=noendperiod]{scrartcl}

\usepackage{amsmath,amssymb}
\usepackage{iftex}
\ifPDFTeX
  \usepackage[T1]{fontenc}
  \usepackage[utf8]{inputenc}
  \usepackage{textcomp} % provide euro and other symbols
\else % if luatex or xetex
  \usepackage{unicode-math}
  \defaultfontfeatures{Scale=MatchLowercase}
  \defaultfontfeatures[\rmfamily]{Ligatures=TeX,Scale=1}
\fi
\usepackage{lmodern}
\ifPDFTeX\else  
    % xetex/luatex font selection
\fi
% Use upquote if available, for straight quotes in verbatim environments
\IfFileExists{upquote.sty}{\usepackage{upquote}}{}
\IfFileExists{microtype.sty}{% use microtype if available
  \usepackage[]{microtype}
  \UseMicrotypeSet[protrusion]{basicmath} % disable protrusion for tt fonts
}{}
\makeatletter
\@ifundefined{KOMAClassName}{% if non-KOMA class
  \IfFileExists{parskip.sty}{%
    \usepackage{parskip}
  }{% else
    \setlength{\parindent}{0pt}
    \setlength{\parskip}{6pt plus 2pt minus 1pt}}
}{% if KOMA class
  \KOMAoptions{parskip=half}}
\makeatother
\usepackage{xcolor}
\usepackage[margin=1in]{geometry}
\setlength{\emergencystretch}{3em} % prevent overfull lines
\setcounter{secnumdepth}{5}
% Make \paragraph and \subparagraph free-standing
\makeatletter
\ifx\paragraph\undefined\else
  \let\oldparagraph\paragraph
  \renewcommand{\paragraph}{
    \@ifstar
      \xxxParagraphStar
      \xxxParagraphNoStar
  }
  \newcommand{\xxxParagraphStar}[1]{\oldparagraph*{#1}\mbox{}}
  \newcommand{\xxxParagraphNoStar}[1]{\oldparagraph{#1}\mbox{}}
\fi
\ifx\subparagraph\undefined\else
  \let\oldsubparagraph\subparagraph
  \renewcommand{\subparagraph}{
    \@ifstar
      \xxxSubParagraphStar
      \xxxSubParagraphNoStar
  }
  \newcommand{\xxxSubParagraphStar}[1]{\oldsubparagraph*{#1}\mbox{}}
  \newcommand{\xxxSubParagraphNoStar}[1]{\oldsubparagraph{#1}\mbox{}}
\fi
\makeatother


\providecommand{\tightlist}{%
  \setlength{\itemsep}{0pt}\setlength{\parskip}{0pt}}\usepackage{longtable,booktabs,array}
\usepackage{calc} % for calculating minipage widths
% Correct order of tables after \paragraph or \subparagraph
\usepackage{etoolbox}
\makeatletter
\patchcmd\longtable{\par}{\if@noskipsec\mbox{}\fi\par}{}{}
\makeatother
% Allow footnotes in longtable head/foot
\IfFileExists{footnotehyper.sty}{\usepackage{footnotehyper}}{\usepackage{footnote}}
\makesavenoteenv{longtable}
\usepackage{graphicx}
\makeatletter
\newsavebox\pandoc@box
\newcommand*\pandocbounded[1]{% scales image to fit in text height/width
  \sbox\pandoc@box{#1}%
  \Gscale@div\@tempa{\textheight}{\dimexpr\ht\pandoc@box+\dp\pandoc@box\relax}%
  \Gscale@div\@tempb{\linewidth}{\wd\pandoc@box}%
  \ifdim\@tempb\p@<\@tempa\p@\let\@tempa\@tempb\fi% select the smaller of both
  \ifdim\@tempa\p@<\p@\scalebox{\@tempa}{\usebox\pandoc@box}%
  \else\usebox{\pandoc@box}%
  \fi%
}
% Set default figure placement to htbp
\def\fps@figure{htbp}
\makeatother
% definitions for citeproc citations
\NewDocumentCommand\citeproctext{}{}
\NewDocumentCommand\citeproc{mm}{%
  \begingroup\def\citeproctext{#2}\cite{#1}\endgroup}
\makeatletter
 % allow citations to break across lines
 \let\@cite@ofmt\@firstofone
 % avoid brackets around text for \cite:
 \def\@biblabel#1{}
 \def\@cite#1#2{{#1\if@tempswa , #2\fi}}
\makeatother
\newlength{\cslhangindent}
\setlength{\cslhangindent}{1.5em}
\newlength{\csllabelwidth}
\setlength{\csllabelwidth}{3em}
\newenvironment{CSLReferences}[2] % #1 hanging-indent, #2 entry-spacing
 {\begin{list}{}{%
  \setlength{\itemindent}{0pt}
  \setlength{\leftmargin}{0pt}
  \setlength{\parsep}{0pt}
  % turn on hanging indent if param 1 is 1
  \ifodd #1
   \setlength{\leftmargin}{\cslhangindent}
   \setlength{\itemindent}{-1\cslhangindent}
  \fi
  % set entry spacing
  \setlength{\itemsep}{#2\baselineskip}}}
 {\end{list}}
\usepackage{calc}
\newcommand{\CSLBlock}[1]{\hfill\break\parbox[t]{\linewidth}{\strut\ignorespaces#1\strut}}
\newcommand{\CSLLeftMargin}[1]{\parbox[t]{\csllabelwidth}{\strut#1\strut}}
\newcommand{\CSLRightInline}[1]{\parbox[t]{\linewidth - \csllabelwidth}{\strut#1\strut}}
\newcommand{\CSLIndent}[1]{\hspace{\cslhangindent}#1}

\usepackage{booktabs}
\usepackage{longtable}
\usepackage{array}
\usepackage{multirow}
\usepackage{wrapfig}
\usepackage{float}
\usepackage{colortbl}
\usepackage{pdflscape}
\usepackage{tabu}
\usepackage{threeparttable}
\usepackage{threeparttablex}
\usepackage[normalem]{ulem}
\usepackage{makecell}
\usepackage{xcolor}
\usepackage[default]{sourcesanspro}
\usepackage{sourcecodepro}
\usepackage{lineno}
\usepackage{setspace}
\doublespacing
\linenumbers
\KOMAoption{captions}{tableheading}
\makeatletter
\@ifpackageloaded{caption}{}{\usepackage{caption}}
\AtBeginDocument{%
\ifdefined\contentsname
  \renewcommand*\contentsname{Table of contents}
\else
  \newcommand\contentsname{Table of contents}
\fi
\ifdefined\listfigurename
  \renewcommand*\listfigurename{List of Figures}
\else
  \newcommand\listfigurename{List of Figures}
\fi
\ifdefined\listtablename
  \renewcommand*\listtablename{List of Tables}
\else
  \newcommand\listtablename{List of Tables}
\fi
\ifdefined\figurename
  \renewcommand*\figurename{Fig.}
\else
  \newcommand\figurename{Fig.}
\fi
\ifdefined\tablename
  \renewcommand*\tablename{Table}
\else
  \newcommand\tablename{Table}
\fi
}
\@ifpackageloaded{float}{}{\usepackage{float}}
\floatstyle{ruled}
\@ifundefined{c@chapter}{\newfloat{codelisting}{h}{lop}}{\newfloat{codelisting}{h}{lop}[chapter]}
\floatname{codelisting}{Listing}
\newcommand*\listoflistings{\listof{codelisting}{List of Listings}}
\makeatother
\makeatletter
\makeatother
\makeatletter
\@ifpackageloaded{caption}{}{\usepackage{caption}}
\@ifpackageloaded{subcaption}{}{\usepackage{subcaption}}
\makeatother

\usepackage{bookmark}

\IfFileExists{xurl.sty}{\usepackage{xurl}}{} % add URL line breaks if available
\urlstyle{same} % disable monospaced font for URLs
\hypersetup{
  colorlinks=true,
  linkcolor={blue},
  filecolor={Maroon},
  citecolor={Blue},
  urlcolor={Blue},
  pdfcreator={LaTeX via pandoc}}


\author{}
\date{}

\begin{document}


\textbf{Saturating allometric relationships reveal how wood density
shapes global tree architecture}

\[ \]

Thi Duyen Nguyen\textsuperscript{1,2}, Masatoshi
Katabuchi\textsuperscript{1}

\[ \]

\textsuperscript{1} CAS Key Laboratory of Tropical Forest Ecology,
Xishuangbanna Tropical Botanical Garden, Chinese Academy of Sciences,
Menglun, Yunnan 666303, China

\textsuperscript{2} University of Chinese Academy of Sciences, Beijing
100049, China

\[ \]

\textbf{Corresponding Authors}:

Masatoshi Katabuchi

E-mail: katabuchi@xtbg.ac.cn; mattocci27@gmail.com

\[ \]

Thi Duyen Nguyen: \url{https://orcid.org/0009-0003-5296-9569}

Masatoshi Katabuchi: \url{https://orcid.org/0000-0001-9900-9029}

\section*{Abstract}\label{abstract}
\addcontentsline{toc}{section}{Abstract}

Allometric equations are fundamental tools in ecological research and
forestry management, widely used for estimating above-ground biomass and
production, serving as the core foundations of dynamic vegetation
models. Using global datasets from Tallo---a tree allometry and crown
architecture database encompassing thousands of species---and TRY---a
plant traits database---we fit Bayesian hierarchical models with three
alternative functional forms (power-law, generalized Michaelis--Menten
(gMM), and Weibull) to characterize how diameter at breast height (DBH),
tree height (H), and crown radius (CR) scale with and without wood
density as a species-level predictor. Our analysis revealed that the
saturating Weibull function best captured the relationship between tree
height and DBH in both functional groups, while the CR-DBH relationship
was best predicted by a power-law function in angiosperms and the gMM
function in gymnosperms. Although including wood density did not
significantly improve predictive performance, it revealed important
ecological trade-offs: lighter-wood angiosperms achieve taller mature
heights more rapidly, and denser wood promotes wider crown expansion
across clades. We also found that accurately estimating DBH required
considering both height and crown size, highlighting how these variables
together distinguish trees of similar height but differing trunk
diameters. Our results emphasize the importance of applying saturating
functions for large trees to improve forest biomass estimates and show
that wood density, while not always predictive at broad scales,
illuminates the biomechanical and ecological constraints underlying
diverse tree architectures. These findings offer practical pathways for
integrating height- and crown-based metrics into existing carbon
monitoring programs worldwide.

\textbf{Keywords:} above ground biomass, crown radius, diameter at
breast height, tree allometry model, tree height, wood density

\section{Introduction}\label{introduction}

Mapping terrestrial carbon stocks is crucial to successfully
implementing climate change mitigation programs
(\citeproc{ref-Chave2014}{Chave et al. 2014}). The accuracy of carbon
stock estimates depends on the availability of reliable allometric
models to deduce the aboveground biomass (AGB) of trees from census data
(\citeproc{ref-Chave2014}{Chave et al. 2014}). Selecting an appropriate
allometric model is, therefore, a critical step in estimating forest
carbon biomass (e.g., \citeproc{ref-Chave2014}{Chave et al. 2014};
\citeproc{ref-Molto2014}{Molto et al. 2014}). Tree allometry describes
the correlations between different biometric measurements of trees, such
as diameter at breast height (DBH), tree height, and tree crown size
(e.g., \citeproc{ref-Shinozaki1964a}{Shinozaki et al. 1964};
\citeproc{ref-Niklas1994}{Niklas 1994}; \citeproc{ref-West1999}{West et
al. 1999}). These relationships are widely utilized to estimate tree
architecture from DBH and to estimate forest biomass and production,
serving as fundamental components of dynamic vegetation models (e.g.,
\citeproc{ref-Feldpausch2011}{Feldpausch et al. 2011};
\citeproc{ref-Falster2017}{Falster et al. 2017};
\citeproc{ref-Song2024}{Song et al. 2024}). Despite significant
progress, developing precise allometric models across regions
characterized by diverse tree species composition and varying forest
structures remains a persistent challenge. This complexity highlights
the need for improved global-scale allometric models that capture the
diversity of tree forms and ecological contexts.

Identifying the most appropriate mathematical representation of
allometric relationships is a subject of ongoing debate. While numerous
papers have used various functional forms for allometric equations
(e.g., \citeproc{ref-Thomas1996}{Thomas 1996};
\citeproc{ref-Feldpausch2011}{Feldpausch et al. 2011};
\citeproc{ref-MartinezCano2019}{Martínez Cano et al. 2019};
\citeproc{ref-Song2024}{Song et al. 2024}), power-law functions remain
the most common choice for describing the allometric scaling of tree
size variables, including height and crown size with DBH (e.g.,
\citeproc{ref-Sileshi2023}{Sileshi et al. 2023};
\citeproc{ref-Jucker2024}{Jucker et al. 2024}). The relationships among
tree size variables often involve complex interactions influenced by
various factors, such as species diversity, environmental conditions,
and individual variability (\citeproc{ref-Laurans2024}{Laurans et al.
2024}; e.g., \citeproc{ref-Poorter2012}{\textbf{Poorter2012?}}).
Consequently, power-law models often fail to accurately capture the
allometric relationships across the full range of tree sizes, leading to
underestimation of dimensions for smaller trees and underestimation for
larger trees (e.g., \citeproc{ref-Fayolle2016}{Fayolle et al. 2016};
\citeproc{ref-Song2024}{Song et al. 2024}). This highlights the need to
adopt alternative functional forms---such as saturating
relationships---that can more accurately reflect the complexity of
allometric scaling in vegetation models (e.g.,
\citeproc{ref-Goodman2014}{Goodman et al. 2014};
\citeproc{ref-Song2024}{Song et al. 2024}).

Earth observation (EO) technologies, including remote sensing and Light
Detection and Ranging (LiDAR), have revolutionized monitoring forests
(e.g., \citeproc{ref-Jucker2017}{Jucker et al. 2017}) by enabling the
identification and measurement of individual tree heights and crown
dimensions using airborne imagery (e.g.,
\citeproc{ref-Shendryk2016}{Shendryk et al. 2016};
\citeproc{ref-Jucker2017}{Jucker et al. 2017}). Because existing
allometric models still depend on stem diameter as a crucial input for
biomass estimation, it also presents the challenge of determining the
most effective way to utilize EO technologies for AGB estimation
(\citeproc{ref-Shendryk2016}{Shendryk et al. 2016};
\citeproc{ref-Jucker2017}{Jucker et al. 2017}). While ground-based LiDAR
can measure DBH directly and is much more efficient than traditional
ground surveys (\citeproc{ref-Tatsumi2023}{Tatsumi et al. 2023}), it is
still less common and convenient than aerial LiDAR. Alternative
approaches focusing on tree height and crown dimensions as model
predictors are also necessary for remote sensing applications
(\citeproc{ref-Jucker2017}{Jucker et al. 2017}). One potential solution
is to estimate DBH from tree height and crown size and then use
established biomass equations, which could be an effective strategy for
integrating airborne imagery into carbon monitoring programs
(\citeproc{ref-Jucker2017}{Jucker et al. 2017}).

Accurately representing tree allometry relationships also requires
considering underlying ecological constraints, particularly the
interplay of plant functional traits and their trade-offs influences
individual tree architecture and the composition of entire forest
communities (e.g., \citeproc{ref-Iida2012}{Iida et al. 2012};
\citeproc{ref-LoubotaPanzou2018}{Loubota Panzou et al. 2018};
\citeproc{ref-Yang2023}{Yang and Swenson 2023}). Under closed-canopy
conditions, strong light competition makes the balance between vertical
and horizontal space use crucial for species coexistence (e.g.,
\citeproc{ref-Aiba2009}{Aiba and Nakashizuka 2009};
\citeproc{ref-Iida2012}{Iida et al. 2012}). Among functional traits,
wood density is closely linked to both plant life history strategies
(\citeproc{ref-Fajardo2022}{\textbf{Fajardo2022?}}) and light demand
(e.g., \citeproc{ref-King2006}{King et al. 2006};
\citeproc{ref-Aiba2009}{Aiba and Nakashizuka 2009};
\citeproc{ref-Poorter2012}{\textbf{Poorter2012?}}). Several studies
suggest that wood density explains a trade-off between rapid stem
expansion and broader crowns, shaping mechanical constraints on species
with diverse architecture and light capture strategies
(\citeproc{ref-Aiba2009}{Aiba and Nakashizuka 2009};
\citeproc{ref-Iida2012}{Iida et al. 2012}; \citeproc{ref-Yang2023}{Yang
and Swenson 2023}). However, these types of studies have examined this
relationship post hoc, relating species-specific allometric parameters
to wood density and risking Type I or Type II errors if the hierarchical
data structure is overlooked (\citeproc{ref-Gelman2006}{Gelman and Hill
2006}). One exception is Martínez Cano et al.
(\citeproc{ref-MartinezCano2019}{2019}), who incorporated wood density
into a hierarchical model of tropical trees and concluded it offered
minimal explanatory power for interspecific variation. Therefore, the
role of wood density in global allometric scaling remains insufficiently
explored.

In this study, we present a robust Bayesian hierarchical approach to
delineate allometric relationships involving DBH, tree height (H), crown
radius (CR), and their interspecific variations. Using large datasets
from the Tallo and TRY databases (\citeproc{ref-Kattge2020}{Kattge et
al. 2020}; \citeproc{ref-Jucker2022}{Jucker et al. 2022}), which span a
diverse array of tree forms across global forest types and climatic
conditions, we overcome the limitations of previous studies that often
relied on site-specific or regional observations with limited samples
and species. By integrating species-level wood density into a multilevel
model, our methodology avoids the post-hoc correlations used in previous
studies, offering a more universal and scalable model, effectively
capturing variability from the species to the community level.
Specifically, we ask the following questions: (1) What is the best
function to estimate tree architecture from DBH and \emph{vice versa}?
(2) How does wood density shape interspecific variation in these
allometric scaling relationships? (3) How does the choice of allometric
relationships affect subsequent AGB estimation?

\section{Materials and Methods}\label{materials-and-methods}

\subsection{Data}\label{data}

Our analysis combined tree allometry data from Tallo, which includes
standardized measurements of DBH, height, crown radius for 5,163 species
across various biomes (\citeproc{ref-Jucker2022}{Jucker et al. 2022},
https://doi.org/10.5281/zenodo.6637599) along with wood density data for
10,995 species from TRY version 6 (\citeproc{ref-Kattge2020}{Kattge et
al. 2020}, https://www.try-db.org/TryWeb/Home.php) (see Supplementary
Table S1 for further details). We thoroughly processed the Tallo dataset
by removing 998 outliers following Jucker et al.
(\citeproc{ref-Jucker2022}{2022}). To ensure consistency, we selected
only species with standardized wood density values in g
cm\textsuperscript{-3} from TRY and removed any records with missing
data. Subsequently, we created an overlapping dataset between Tallo and
TRY, resulting in a combined dataset encompassing allometric and wood
density information for 355,881 individuals across 2,961 species.

To reduce computational complexity and ensure the robustness of the
models, we further filtered species with fewer than 20 individuals and
randomly sampled up to 100 individuals per species. This step was
necessary because the dataset is unbalanced; for example, \emph{Quercus
ilex} alone accounts for over 20,000 individuals, representing
approximately 10\% of the dataset. However, \emph{Quercus ilex} is not
necessarily the most abundant species globally. Its prevalence in this
dataset is a result of sampling and does not reflect its true global
abundance. By limiting the number of individuals per species, we
minimize bias due to data imbalance and prevent the overrepresentation
of any single abundant species (\citeproc{ref-Gelman2006}{Gelman and
Hill 2006}), thereby ensuring that the model remains broadly applicable.
Since not all trees have a complete set of information, we created
specific sub-datasets to select the best predictive model for each
allometric relationship for two clades (i.e., angiosperms and
gymnosperms; see below and Supplementary Table S2 for more details).

\subsection{Tree allometry models}\label{tree-allometry-models}

We developed hierarchical Bayesian models to estimate the allometric
relationships between tree architectural variables, including DBH (cm),
tree height (H, m), crown radius (CR, m), and wood density (g
cm\textsuperscript{-3}). We assumed a Gaussian likelihood (\(N\)) for
the natural logarithm of the response variable, \(y_{(i,j)}\)
representing either tree height, crown radius, or DBH for each tree
\(i\) in species \(j\):

\begin{equation}\phantomsection\label{eq-like}{
\log(y_{i,j}) \sim N(f(X_{i,j}, \pmb{\beta}_{j}), \sigma_y),
}\end{equation}

where \(f(\cdot)\) predicts the expected natural log tree height or
crown radius from observed DBH (and \emph{vice versa}),
\(\pmb{\beta}_{j}=\{a_j, b_j, k_j\}\) denotes the vector of
species-specific parameters, and \(\sigma_y\) represents the standard
deviation of the observation error. We considered three functional forms
for the relationship between tree allometric variables
(\citeproc{ref-Sullivan2018}{Sullivan et al. 2018};
\citeproc{ref-MartinezCano2019}{Martínez Cano et al. 2019}):

Power-law: \begin{equation}\phantomsection\label{eq-pl}{
y = aX^b,
}\end{equation}

Generalized Michaelis-Menten (gMM):
\begin{equation}\phantomsection\label{eq-gMM}{
y = \frac{aX^b}{k + X^b},
}\end{equation}

Weibull: \begin{equation}\phantomsection\label{eq-wb}{
y = a\left\{1-\mathrm{exp}(-bX^k)\right\},
}\end{equation}

The power-law model (Eq.~\ref{eq-pl}) is one of the simplest and most
commonly used functions to describe biological scaling relationships
(e.g., \citeproc{ref-West1999}{West et al. 1999};
\citeproc{ref-Brown2004}{Brown et al. 2004}). We also tested two
nonlinear models, the generalized Michaelis--Menten (gMM)
(Eq.~\ref{eq-gMM}) and the Weibull function (Eq.~\ref{eq-wb}), to
account for a saturating relationship between tree dimensions and DBH
(e.g., \citeproc{ref-MartinezCano2019}{Martínez Cano et al. 2019}).

We applied the power-law model to estimate tree height and crown radius
from DBH, as well as DBH from tree height and/or crown radius.
Specifically, when predicting DBH from height and/or crown radius, we
tested four formulations: (1) using CR and H as separate factors,
\(\mathrm{DBH} = a_1 \mathrm{CR}^{b_1} \mathrm{H}^{c_1}\), denoted as
\(CR,H\); (2) using CR and H as a compound term,
\(\mathrm{DBH} = a_2 (\mathrm{CR} \times \mathrm{H})^{b_2}\), denoted as
\(CR \times H\); (3) using only height,
\(\mathrm{DBH} = a_3 \mathrm{H}^{b_3}\); and (4) using only crown
radius, \(\mathrm{DBH} = a_4 \mathrm{CR}^{b_4}\). In contrast, the gMM
and Weibull models were only used to estimate tree height and crown
radius from DBH. Because these two models specifically describe
saturating relationships (Eq.~\ref{eq-gMM}; Eq.~\ref{eq-wb}), their
formulation does not allow simply switching the roles of explanatory and
response variables.

To compare the performance of different models for the same response
variable, we used Pareto-smoothed importance sampling leave-one-out
cross-validation (PSIS-LOO; Vehtari et al.
(\citeproc{ref-Vehtari2017}{2017})). PSIS-LOO is an accurate and
reliable approximation to standard leave-one-out cross-validation (LOO),
which is a robust method for comparing models with different numbers of
parameters (\citeproc{ref-Vehtari2017}{Vehtari et al. 2017}). We
utilized PSIS-LOO to calculate the LOO Information Criterion (LOOIC) for
each model. A model with lower LOOIC is considered to have better
predictive accuracy (\citeproc{ref-Vehtari2017}{Vehtari et al. 2017}).

Once the best-performing models were identified for each allometric
relationship, we then tested whether adding a species-level predictor
(wood density) would further improve that model's predictive
performance. Species-specific parameters (\(\beta_j\)) were modeled as a
function of wood density (\(u_j\)):
\begin{equation}\phantomsection\label{eq-wood}{
\beta_{j} \sim N(\gamma_{0} + \gamma_{1} u_j, \sigma_{\beta}),
}\end{equation}

where \(\gamma_{0}\) and \(\gamma_{1}\) are the species-level intercept
and slope, respectively, and \(\sigma_{\beta}\) is the standard
deviation of the species-specific parameters. Additionally, we explored
a model that accounts for covariance between species-specific parameters
using a multivariate normal distribution. However, due to the complexity
of this approach, the model did not converge for the nonlinear models,
and thus, we did not include it in the final analysis.

Posterior distributions of all parameters were estimated using the
No-U-Turn Sampler (NUTS), an adaptive variant of the Hamiltonian Monte
Carlo (HMC), algorithm implemented in Stan
(\citeproc{ref-Carpenter2017}{Carpenter et al. 2017}) using
weakly-informative priors (\citeproc{ref-Gelman2008}{Gelman et al.
2008}). The HMC algorithm uses gradient information to propose new
states in the Markov chain, leading to a more efficient exploration of
the target distribution than traditional Markov chain Monte Carlo (MCMC)
methods that rely on random proposals
(\citeproc{ref-Carpenter2017}{Carpenter et al. 2017}). This efficiency
allows us to achieve convergence with fewer iterations than traditional
MCMC methods. Inferences were based on 1,000 posterior samples following
1,000 warmup iterations for three parallel chains. Convergence was
assessed using the Gelman--Rubin statistic with a convergence threshold
of 1.1 (\citeproc{ref-Gelman2013}{Gelman et al. 2015}) and ensuring
effective sample sizes greater than 300
(\citeproc{ref-Vehtari2021}{Vehtari et al. 2021}). We also reported the
median and 95\% Bayesian credible intervals (CIs) for each parameter
(Supplementary Tables S3-S8). Estimates were considered significant when
95\% CIs did not overlap with zero.

Statistical analyses were performed in R version 4.3.2
(\citeproc{ref-RCoreTeam2023}{R Core Team 2023}), using the R packages
\emph{targets} version 1.7.0 for workflow management
(\citeproc{ref-Landau2021}{Landau 2021}). Codes are made available on
GitHub at \url{https://github.com/duyenecology/tree-allometry} (Stan
code is provided in Supplementary Section S2).

\subsection{Aboveground biomass
estimates}\label{aboveground-biomass-estimates}

For angiosperm species, we estimated aboveground biomass (AGB; kg of dry
mass) using the equation from Chave et al.
(\citeproc{ref-Chave2014}{2014}):

\begin{equation}\phantomsection\label{eq-agb}{
\mathrm{AGB} = 0.0673 \times (\rho D^2H)^{0.976}
}\end{equation}

where \(\rho\) is wood density (g cm\textsuperscript{-3}), \(D\) (cm) is
DBH, and \(H\) (m) is tree height. We either used both observed DBH and
height or substituted one of these variables with an estimate from our
allometric models, allowing us to assess how different allometric
relationships affect AGB calculations. Only trees with a DBH ≥ 5 cm were
used, as those below this threshold hold a small fraction of forest
carbon stocks and were excluded from the calibration of Chave et al.
(\citeproc{ref-Chave2014}{2014})'s equation. Additionally, the
sub-datasets used to select the optimal predictive model for each
allometric relationship differed as not all trees had a complete
information set (see above and Supplementary Table S2). Consequently,
the final dataset for estimating AGB in angiosperms includes 124,206
trees with sufficient allometric information (i.e., DBH, H, CR, WD)
across 624 species. We did not extend this approach to gymnosperms
because no general AGB equation is available for them, and most existing
equations are species-specific and do not incorporate wood density
(e.g., \citeproc{ref-Moore2010}{Moore 2010};
\citeproc{ref-Bazrgar2024}{Bazrgar et al. 2024}).

\section{Results}\label{results}

\subsection{Tree height allometry}\label{tree-height-allometry}

The optimal model to predict tree height was a Weibull function for both
angiosperms (ang) and gymnosperms (gym) (Table 1). Across the global
dataset, the best model to predict H from DBH of each division was:
\begin{equation}\phantomsection\label{eq-h_ang}{
\mathrm{H_{ang}} =50.0
\left\{ 1 - \exp\left( - 0.0476
\mathrm{DBH}^{0.629}
\right) \right\}
}\end{equation}

\begin{equation}\phantomsection\label{eq-h_gym}{
\mathrm{H_{gym}} = 117.4
\left\{ 1 - \exp\left( - 0.012
\mathrm{DBH}^{0.750}
\right) \right\}.
}\end{equation}

The posterior medians with their 95\% CIs were as follows: asymptote
\(a_{\text{ang}}\) = 50.0 {[}48.0, 52.2{]}, \(a_{\text{gym}}\) = 117.4
{[}89.22, 175.1{]}; exponent \(b_{\text{ang}}\) = 0.0476 {[}0.0458,
0.0495{]}, \(b_{\text{gym}}\) = 0.012 {[}0.008, 0.016{]}; and shape
parameter \(k_{\text{ang}}\) = 0.629 {[}0.618, 0.639{]},
\(k_{\text{gym}}\) = 0.750 {[}0.719, 0.784{]}.

The second-best model, which used the gMM function, showed poorer
predictive performance in angiosperms (\(\Delta\) LOOIC = 101.01) but
showed competitive predictive performance in gymnosperms (\(\Delta\)
LOOIC = 4.87). While power-law models (Eq.~\ref{eq-pl}) can exhibit
saturating behavior when 0 \textless{} \emph{b} \textless{} 1, they
performed the worst overall compared to other saturating models
(Eq.~\ref{eq-gMM}; Eq.~\ref{eq-wb}). Incorporating wood density did not
significantly improve predictive performance (Table 1).

\textbf{Table 1}: Summary of the results of the model selection
procedure. Models were ranked using the Leave-One-Out Information
Criterion (LOOIC) to assess their predictive performance. For each
model, we report the Expected Log Predictive Density (ELPD), the LOOIC,
and the \(R^2\) values. The difference in LOOIC relative to the
best-performing model, \(\Delta\)\_LOOIC, is also included. Lower
\(\Delta\)\_LOOIC values indicate models that perform closer to the best
model in terms of predictive accuracy, while higher ELPD values reflect
better model fit. Higher \(R^2\) values reflect better explanatory
power, indicating how well the model explains the variance in the
dependent variable.

\newpage

\begingroup\fontsize{10}{12}\selectfont

\begin{longtable*}[t]{ccccccccc}
\toprule
Dependent variable & Predictor variable & Division & Functional form & Wood density & ELPD & LOOIC & ΔLOOIC & R2\\
\midrule
Tree Height & DBH & Angiosperm & Weibull & With & -4317.73 & 8635.46 & 0.00 & 0.823\\
Tree Height & DBH & Angiosperm & Weibull & Without & -4318.55 & 8637.10 & 1.65 & 0.823\\
Tree Height & DBH & Angiosperm & gMM & Without & -4368.23 & 8736.47 & 101.01 & 0.825\\
Tree Height & DBH & Angiosperm & Power-Law & Without & -4720.21 & 9440.42 & 804.96 & 0.813\\
Tree Height & DBH & Gymnosperm & Weibull & Without & -540.02 & 1080.04 & 0.00 & 0.859\\
\addlinespace
Tree Height & DBH & Gymnosperm & Weibull & With & -540.37 & 1080.74 & 0.71 & 0.859\\
Tree Height & DBH & Gymnosperm & gMM & Without & -542.45 & 1084.90 & 4.87 & 0.858\\
Tree Height & DBH & Gymnosperm & Power-Law & Without & -581.04 & 1162.08 & 82.05 & 0.854\\
Crown Radius & DBH & Angiosperm & Power-Law & Without & -20264.58 & 40529.16 & 0.00 & 0.724\\
Crown Radius & DBH & Angiosperm & Power-Law & With & -20265.11 & 40530.22 & 1.07 & 0.724\\
\addlinespace
Crown Radius & DBH & Angiosperm & Weibull & Without & -20326.86 & 40653.73 & 124.57 & 0.722\\
Crown Radius & DBH & Angiosperm & gMM & Without & -20327.75 & 40655.50 & 126.35 & 0.722\\
Crown Radius & DBH & Gymnosperm & gMM & Without & -811.34 & 1622.68 & 0.00 & 0.728\\
Crown Radius & DBH & Gymnosperm & gMM & With & -813.38 & 1626.77 & 4.09 & 0.728\\
Crown Radius & DBH & Gymnosperm & Weibull & Without & -815.13 & 1630.26 & 7.58 & 0.728\\
\addlinespace
Crown Radius & DBH & Gymnosperm & Power-Law & Without & -892.70 & 1785.40 & 162.72 & 0.712\\
DBH & CR, H & Angiosperm & Power-Law & With & -15613.32 & 31226.64 & 0.00 & 0.872\\
DBH & CR, H & Angiosperm & Power-Law & Without & -15615.02 & 31230.05 & 3.41 & 0.803\\
DBH & CR × H & Angiosperm & Power-Law & Without & -17144.13 & 34288.25 & 3061.61 & 0.855\\
DBH & H & Angiosperm & Power-Law & Without & -21457.80 & 42915.59 & 11688.96 & 0.825\\
\addlinespace
DBH & CR & Angiosperm & Power-Law & Without & -25578.55 & 51157.11 & 19930.47 & 0.728\\
DBH & CR, H & Gymnosperm & Power-Law & Without & -961.30 & 1922.59 & 0.00 & 0.854\\
DBH & CR, H & Gymnosperm & Power-Law & With & -961.55 & 1923.10 & 0.51 & 0.897\\
DBH & CR × H & Gymnosperm & Power-Law & Without & -1136.13 & 2272.27 & 349.67 & 0.885\\
DBH & H & Gymnosperm & Power-Law & Without & -2000.82 & 4001.63 & 2079.04 & 0.809\\
\addlinespace
DBH & CR & Gymnosperm & Power-Law & Without & -2392.47 & 4784.95 & 2862.36 & 0.752\\
\bottomrule
\end{longtable*}
\endgroup{}

\newpage

\begin{figure}

\centering{

\pandocbounded{\includegraphics[keepaspectratio]{../figs/h_cr_dbh_non_log.png}}

}

\caption{\label{fig-compare}Allometric relationships between tree height
and crown radius with DBH of angiosperm (a, c) and gymnosperm (b, d)
based on three different models: power-law (orange), generalized
Michaelis-Menten (green), and Weibull (blue). The density of overlapping
points is shown using a color gradient ranging from black (low density)
to yellow (high density). Point density was calculated using a 2D kernel
density estimation on a 300 \(\times\) 300 grid, estimating the
concentration of data points in the space of DBH and tree height or
crown radius (see Supplementary Figure S1 for the log-scale version).}

\end{figure}%

\subsection{Crown Radius allometry}\label{crown-radius-allometry}

The power-law function (0 \textless{} \emph{b} \textless{} 1) provided
the best predictive performance for modeling the relationship between
crown radius and DBH in angiosperms. Whereas for gymnosperms, the gMM
function offered the best performance for this allometric scaling (Table
1). Across the global dataset, the best model for predicting CR from DBH
of each division was: \begin{equation}\phantomsection\label{eq-cr_ang}{
\mathrm{CR_{ang}} = \mathrm{0.349}
\, \mathrm{DBH}^{\mathrm{0.626}}
}\end{equation}

\begin{equation}\phantomsection\label{eq-cr_gym}{
\mathrm{CR_{gym}} = \frac{\mathrm{769.4}
\, \mathrm{DBH}^{\mathrm{0.631}}}
{\mathrm{2657}
+ \mathrm{DBH}^{\mathrm{0.631}}}.
}\end{equation}

The posterior medians with their 95\% CIs were as follows: scaling
factor \(a_{\text{ang}}\) = 0.349 {[}0.340, 0.359{]}, asymptote
\(a_{\text{gym}}\) = 769.4 {[}227.3, 1818{]}; exponent
\(b_{\text{ang}}\) = 0.626 {[}0.617, 0.635{]}, \(b_{\text{gym}}\) =
0.631 {[}0.600, 0.663{]}; and scaling constant \(k_{\text{gym}}\) = 2657
{[}808.7, 6274{]}.

For angiosperms, the saturating functions (Eq.~\ref{eq-gMM};
Eq.~\ref{eq-wb}) performed worse than the power-law model, with
\(\Delta\) LOOIC values of 124.57 for gMM and 126.35 for the Weibull
function. In gymnosperms, the Weibull function showed a competitive
performance compared to the gMM function, while the power-law function
(Eq.~\ref{eq-pl}) performed the worst, with a \(\Delta\) LOOIC of
162.72. Incorporating wood density did not significantly improve
predictive performance (Table 1).

\textbf{Table 2}: Posterior estimates of the parameters for the best
hierarchical models selected for tree height, crown radius, and DBH
allometries. Each set includes the best models for angiosperms and
gymnosperms, with posterior estimates reported both without and with the
inclusion of wood density. Table entries represent the median and 95\%
Bayesian credible intervals for the community-level parameters of each
allometric function. For tree height, the Weibull model was identified
as the best for both angiosperm and gymnosperm groups. Crown radius was
best described by a power-law model for angiosperms and a gMM model for
gymnosperms. The best model for DBH allometry in both divisions
incorporated both tree height and crown radius information (\(CR,H\)
model).

\newpage

\begingroup\fontsize{10}{12}\selectfont

\begin{longtable*}[t]{cccccccc}
\toprule
Dependent\_variable & Division & Functional\_form & Wood\_density & Parameter & Intercept\_CI & Slope\_CI & Tau\_CI\\
\midrule
Tree Height & Angiosperm & Weibull & With & a & 50.0358625836953 (48.0185910757846, 52.2782021867477) & 1.0030357535955 (0.966279786642521, 1.04204878932111) & 0.1094080515 (0.08504578, 0.133867675)\\
Tree Height & Angiosperm & Weibull & With & b & 0.0476521 (0.0456004375, 0.04954112) & -0.0002497295 (-0.0018876475, 0.001416113) & 0.0117625713666667 (0.0108990775, 0.01264796)\\
Tree Height & Angiosperm & Weibull & With & k & 0.627988 (0.6179271, 0.638469025) & -0.0155465 (-0.025364455, -0.005908563) & 0.141325601 (0.134013175, 0.148615)\\
Tree Height & Angiosperm & Weibull & Without & a & 49.986601524105 (48.0397359146718, 52.1676071716969) & - & 0.108623839 (0.084699285, 0.13259475)\\
Tree Height & Angiosperm & Weibull & Without & b & 0.04761845 (0.045788395, 0.0495194425) & - & 0.0117794441333333 (0.0109647775, 0.012630005)\\
\addlinespace
Tree Height & Angiosperm & Weibull & Without & k & 0.628505 (0.61806375, 0.63871895) & - & 0.142012394 (0.135037575, 0.1493971)\\
Tree Height & Gymnosperm & Weibull & With & a & 114.392440818366 (85.1394073936058, 170.399810107161) & 0.973669651215167 (0.775709235915639, 1.21272840336665) & 0.0976113393243333 (0.00465741875, 0.256888)\\
Tree Height & Gymnosperm & Weibull & With & b & 0.01226515 (0.00835557525, 0.01627371) & 0.0006333235 (-0.001878769, 0.003421633) & 0.00382281789033333 (0.0022539225, 0.00552581075)\\
Tree Height & Gymnosperm & Weibull & With & k & 0.75131 (0.720762275, 0.78392225) & -0.0223577 (-0.053752515, 0.00762075675000001) & 0.11290265 (0.0904752175, 0.139776975)\\
Tree Height & Gymnosperm & Weibull & Without & a & 117.410929580669 (89.2163884742527, 175.128720630186) & - & 0.0912130060437333 (0.00357252325, 0.229208275)\\
\addlinespace
Tree Height & Gymnosperm & Weibull & Without & b & 0.01189855 (0.00806481775, 0.0155006175) & - & 0.00370917828 (0.00230250075, 0.005250409)\\
Tree Height & Gymnosperm & Weibull & Without & k & 0.750261 (0.71890345, 0.7840528) & - & 0.113927871366667 (0.0919335325, 0.139492325)\\
Crown Radius & Angiosperm & Power-Law & With & a & 0.348870320614843 (0.339558002412552, 0.358617649941404) & 0.067951 (0.0395029625, 0.09696407) & 0.331462941666667 (0.3107583, 0.353737675)\\
Crown Radius & Angiosperm & Power-Law & With & b & 0.6262655 (0.61724685, 0.6347153) & -0.0110058 (-0.01988032, -0.00253456499999999) & 0.0921479955666667 (0.0844064425, 0.100138025)\\
Crown Radius & Angiosperm & Power-Law & Without & a & 0.349041308960668 (0.340277352929455, 0.358866706688696) & - & 0.336416977 (0.315543425, 0.35812015)\\
\addlinespace
Crown Radius & Angiosperm & Power-Law & Without & b & 0.625775 (0.61686365, 0.634504125) & - & 0.0928698366666667 (0.08518862, 0.101327225)\\
Crown Radius & Gymnosperm & gMM & With & a & 975.002465780217 (311.898479923573, 2010.23216629045) & 0.98826173732258 (0.765945182590572, 1.26629969928617) & 0.366709977 (0.27977655, 0.473532775)\\
Crown Radius & Gymnosperm & gMM & With & b & 0.629287 (0.599257875, 0.658672975) & 0.03446325 (0.00506576425, 0.0642449925) & 0.0948956725666667 (0.0691086525, 0.126110225)\\
Crown Radius & Gymnosperm & gMM & With & k & 3371.455 (1094.23225, 6977.7715) & 35.71465 (-768.70475, 913.79515) & 9.363854693469 (0.10023131, 52.3180125)\\
Crown Radius & Gymnosperm & gMM & Without & a & 769.406932080409 (227.345355225331, 1818.30585925327) & - & 0.367348107833333 (0.280439875, 0.465157075)\\
\addlinespace
Crown Radius & Gymnosperm & gMM & Without & b & 0.630821 (0.60042295, 0.6625657) & - & 0.101279463966667 (0.0755376825, 0.130907775)\\
Crown Radius & Gymnosperm & gMM & Without & k & 2657.085 (808.712975, 6274.33575) & - & 18.9842685191737 (0.11158275, 200.190725)\\
DBH & Angiosperm & Power-Law & With & a & 0.948873203444669 (0.92824124982295, 0.970115458681121) & -0.00254114 (-0.026471175, 0.019015775) & 0.301446662333333 (0.28422895, 0.320054325)\\
DBH & Angiosperm & Power-Law & With & b & 0.3733935 (0.36112825, 0.385692275) & -0.003273 (-0.014598435, 0.008914959) & 0.1350893 (0.12519295, 0.145357375)\\
DBH & Angiosperm & Power-Law & With & c & 0.589474 (0.574794525, 0.604623) & 0.005107385 (-0.00957317475, 0.020006805) & 0.167959387666667 (0.1564333, 0.1794579)\\
\addlinespace
DBH & Angiosperm & Power-Law & Without & a & 0.948185804143355 (0.927358177080849, 0.969205798294886) & - & 0.302088741 (0.2843181, 0.320217575)\\
DBH & Angiosperm & Power-Law & Without & b & 0.373565 (0.361136275, 0.38552) & - & 0.134749882666667 (0.125359425, 0.145310525)\\
DBH & Angiosperm & Power-Law & Without & c & 0.589786 (0.5749327, 0.604729725) & - & 0.168123297666667 (0.156878925, 0.180291175)\\
DBH & Gymnosperm & Power-Law & With & a & 0.970943839654932 (0.900241719046287, 1.04481714578926) & -0.04506835 (-0.119699575, 0.0295394575) & 0.273574103 (0.22492635, 0.338502075)\\
DBH & Gymnosperm & Power-Law & With & b & 0.408055 (0.37113765, 0.443653025) & 0.02839555 (-0.007731709, 0.0616446325) & 0.112309440333333 (0.0845160475, 0.14804745)\\
\addlinespace
DBH & Gymnosperm & Power-Law & With & c & 0.5880865 (0.538007975, 0.636716225) & -0.0363728 (-0.0841089975, 0.0139803) & 0.164765638 (0.127992925, 0.2091867)\\
DBH & Gymnosperm & Power-Law & Without & a & 0.972360848599851 (0.903429908240124, 1.05107834246826) & - & 0.272779666666667 (0.22238815, 0.3356855)\\
DBH & Gymnosperm & Power-Law & Without & b & 0.409878 (0.3725057, 0.446499075) & - & 0.114795101166667 (0.0861228225, 0.15017945)\\
DBH & Gymnosperm & Power-Law & Without & c & 0.587609 (0.540198275, 0.638262825) & - & 0.165230531666667 (0.129457475, 0.209437675)\\
\bottomrule
\end{longtable*}
\endgroup{}

\subsection{DBH allometry}\label{dbh-allometry}

The \(CR,H\) model, which treated CR and H as separate factors,
exhibited the best predictive performance among all tested models (Table
1). Across the global dataset, the most effective model for predicting
DBH from H and CR for each division was:
\begin{equation}\phantomsection\label{eq-dbh_ang}{
\mathrm{DBH_{ang}} = \mathrm{0.948}
\mathrm{CR}^{\text{0.374}}
\mathrm{H}^{\text{0.590}}
}\end{equation}

\begin{equation}\phantomsection\label{eq-dbh_gym}{
\mathrm{DBH_{gym}} = \mathrm{0.972}
\mathrm{CR}^{\text{0.410}}
\mathrm{H}^{\text{0.588}}.
}\end{equation}

The posterior medians with their 95\% CIs were as follows: scale
\(a_{\text{ang}}\) = 0.948 {[}0.927, 0.969{]}, \(a_{\text{gym}}\) =
0.972 {[}0.903, 1.051{]}; exponent for CR \(b_{\text{ang}}\) = 0.374
{[}0.361, 0.386{]}, \(b_{\text{gym}}\) = 0.410 {[}0.373, 0.446{]}; and
exponent for H \(c_{\text{ang}}\) = 0.590 {[}0.575, 0.605{]},
\(c_{\text{gym}}\) = 0.588 {[}0.540, 0.638{]}. Incorporating wood
density did not significantly improve predictive performance (Table 1).

\subsection{Wood density and tree architectural
diversity}\label{wood-density-and-tree-architectural-diversity}

Although including wood density did not significantly improve the
predictive performance of any of the allometric relationships, it can
enhance our understanding of the underlying biological mechanisms
driving tree architectural diversity. In H-DBH allometry, the shape
parameter \(k\) in angiosperms showed a significant negative
relationship with wood density, while the asymptote parameter \(a\) and
the scale parameter \(b\) showed no significant associations
(Fig.~\ref{fig-wood1} a, b, c; Fig.~\ref{fig-wood2} a). In gymnosperms,
none of the parameters significantly correlated with wood density
(Fig.~\ref{fig-wood1} d, e, f).

In CR-DBH allometry, angiosperms exhibited a positive relationship
between the scaling factor \(a\) and wood density and a negative
association between the exponent parameter \(b\) and wood density
(Fig.~\ref{fig-wood1} g, h; Fig.~\ref{fig-wood2} b). In gymnosperms, we
found a positive association between wood density and the exponent
parameter \(b\), whereas neither the asymptote parameter \(a\) nor the
scaling constant \(k\) showed significant relationship with this
functional trait (Fig.~\ref{fig-wood1} i, j, k; Fig.~\ref{fig-wood2} c).

\begin{figure}[H]

\centering{

\pandocbounded{\includegraphics[keepaspectratio]{../figs/wd_com.png}}

}

\caption{\label{fig-wood1}Relationships between wood density and
species-specific allometry parameters for tree height (a, b, c, d, e, f)
and crown radius (g, h, i, j, k) of angiosperms (purple) and gymnosperms
(green). Points represent posterior medians of species-specific
parameters, with vertical bars indicating 95\% Bayesian credible
intervals (CIs). Thick solid lines represent significant relationships
across species, with the shaded area illustrating the 95\% CIs, while
dashed lines indicate non-significant relationships.}

\end{figure}%

\begin{figure}[H]

\centering{

\pandocbounded{\includegraphics[keepaspectratio]{../figs/wd_ef.png}}

}

\caption{\label{fig-wood2}Effect of wood density on tree height
allometry in angiosperms (a) and on crown radius allometry in
angiosperms (b) and gymnosperms (c). Curves represent the allometric
scaling for species with light (orange) and dense (brown) wood density,
corresponding to the 10\% and 90\% quantiles of wood density
distribution in the observed dataset, respectively. The shaded areas
illustrate the 95\% CIs. Only statistically significant associations
from Fig.~\ref{fig-wood1} are plotted.}

\end{figure}%

\subsection{Estimating aboveground
biomass}\label{estimating-aboveground-biomass}

AGB estimates based on the Weibull model (\(\text{AGB}_\text{H-WB}\))
showed the highest predictive accuracy based both on the lowest root
mean square error (RMSE) of 0.279 Mg and the smallest bias of 2.043\%
(Fig.~\ref{fig-agb} a). Estimates based on the power-law model
(\(\text{AGB}_\text{H-PL}\)) also showed competitive performance but had
higher RMSE and bias compared to the Weibull-based approach, with RMSE
of 0.28 Mg and bias of 2.29\% (Fig.~\ref{fig-agb} b). In contrast,
models that used estimated DBH performed substantially worse
(Fig.~\ref{fig-agb} c-f).

\begin{figure}[H]

\centering{

\pandocbounded{\includegraphics[keepaspectratio]{../figs/agb_ang.png}}

}

\caption{\label{fig-agb}Comparison of individual tree above-ground
biomass estimated from different approaches with reference AGB
(\(\text{AGB}_\text{ref}\)) derived from observed DBH and tree height in
angiosperms. (a, b) AGB estimates based on observed DBH and tree height
predicted by the Weibull function (\(\text{AGB}_\text{H-WB}\)) or the
power-law function (\(\text{AGB}_\text{H-PL}\)). (c--f) AGB estimates
based on observed tree height and DBH predicted from crown radius and
tree height as separate factors
(\(\text{AGB}_{\text{DBH} \sim \text{CR,H}}\)), as a compound term
(\(\text{AGB}_{\text{DBH} \sim \text{CR × H}}\)), from tree height alone
(\(\text{AGB}_{\text{DBH} \sim \text{H}}\)), or from crown radius alone
(\(\text{AGB}_{\text{DBH} \sim \text{CR}}\)). Dashed lines represent the
1:1 relationships, and the solid lines are linear regression fits to the
data points to highlight how predictive accuracy varies with tree size.
RMSE (root mean square error) and bias values for each model are
reported.}

\end{figure}%

\section{Discussion}\label{discussion}

Relationships among tree-size variables are foundational to both
ecological theory and practical forestry applications, informing
everything from carbon stock estimation to dynamic vegetation modeling.
In this study, we tested multiple functional forms---power-law,
generalized Michaelis--Menten (gMM), and Weibull---within a Bayesian
hierarchical framework to identify which best describes the linkages
between diameter at breast height (DBH), tree height (H), and crown
radius (CR) across a global dataset. Our results reveal that the Weibull
function most reliably captures the saturating nature of the H--DBH
relationship in both angiosperms and gymnosperms, while the CR--DBH
relationship diverges by clade, being well described by a power-law in
angiosperms but requiring a gMM form in gymnosperms. Although
incorporating wood density did not significantly enhance predictive
performance, it illuminated interspecific architectural constraints and
trade-offs, underscoring the complex interplay between mechanical
support, crown expansion, and life history strategies. Below, we discuss
these findings in the context of tree allometry models, the role of
functional traits, and forest biomass estimation.

\subsection{Tree allometry}\label{tree-allometry}

Our result supported a saturating H-DBH relationship in both angiosperm
and gymnosperm species, consistent with theoretical expectations
(\citeproc{ref-Falster2003}{Falster and Westoby 2003};
\citeproc{ref-Niklas2007}{Niklas 2007}) and empirical observations
(e.g., \citeproc{ref-Feldpausch2011}{Feldpausch et al. 2011};
\citeproc{ref-MartinezCano2019}{Martínez Cano et al. 2019};
\citeproc{ref-Song2024}{Song et al. 2024}). A recent study using Tallo
suggested the Weibull function as the best model for six plant
functional types (PFTs) (\citeproc{ref-Song2024}{Song et al. 2024}).
While PFTs-based approaches can illuminate large-scale patterns, they
may obscure species-specific variation in allometric scaling, and
locally abundant species may bias the overall estimates. Our best
predictive model effectively captures species-specific patterns
(Supplementary Table S6) despite underestimating the height of
individuals with DBH \textgreater{} 30 cm due to substantial
interspecific variation in H--DBH allometric scaling
(Fig.~\ref{fig-compare}). This highlights the advantages of a multilevel
modeling approach in addressing systematic differences across species.
Additionally, the metabolic scaling theory (MST) posits logarithmic
scaling relationships among different axes of tree size, with an
exponent close to 2/3 (\(H \propto DBH^{2/3}\),
\(CR \propto DBH^{2/3}\)), based on assumptions regarding mechanical and
hydraulic constraints to plant growth (\citeproc{ref-West1999}{West et
al. 1999}). However, our findings revealed that the power-law model
exponent of angiosperms (\emph{b} = 0.54) was significantly smaller,
while the actual exponent in gymnosperms (\emph{b} = 0.7) was larger
than the theoretical value of 2/3 (Supplementary Table S3). This
divergence highlights the importance of separately modeling the
allometric relationships of each functional group to accurately capture
their distinct scaling patterns. Moreover, the power-law function
performed worse than the saturating functions (Table 1), indicating that
the actual H-DBH relationship does not follow the simple scaling laws
proposed by MST.

Our analysis also revealed divergent patterns in CR-DBH allometry,
highlighting fundamental differences in growth strategies and ecological
adaptations between angiosperms and gymnosperms. In angiosperms, the
CR--DBH relationship follows a power-law form, suggesting a capacity for
plastic crown expansion in competitive environments
(\citeproc{ref-Lines2012}{Lines et al. 2012};
\citeproc{ref-Hulshof2015}{Hulshof et al. 2015}; e.g.,
\citeproc{ref-Poorter2012}{\textbf{Poorter2012?}}), a conclusion
consistent with a study in tropical moist forests by Martínez Cano et
al. (\citeproc{ref-MartinezCano2019}{2019}). In addition, the estimated
exponent aligns closely with MST predictions for crown radius,
supporting the selection of the power-law model for angiosperms, where
the community-level exponent \emph{b} is near the expected 2/3
(\(b_{\text{ang}} = 0.625775 \,
\left[ 0.6168636, 0.6345041 \right]\), Table 2). In contrast, the CR-DBH
allometry in gymnosperms showed an asymptotic relationship best
described by the gMM function, with the Weibull function also showing
competitive predictive performance (Table 1). This asymptotic pattern
likely arises from structural constraints in gymnosperms growth forms
that favour vertical over lateral expansion (e.g.,
\citeproc{ref-Lines2012}{Lines et al. 2012};
\citeproc{ref-Hulshof2015}{Hulshof et al. 2015}), an adaptation that can
mitigate risks such as branch breakage under heavy snow loads (e.g.,
\citeproc{ref-Mikola1938}{Mikola 1938}; \citeproc{ref-Loehle2016}{Loehle
2016}).

Finally, we tested models to estimate DBH from biometric variables
(i.e., tree height and crown radius) measurable via Earth observation
(EO) platforms, such as remote sensing and LiDAR
(\citeproc{ref-Filipescu2012}{Filipescu et al. 2012};
\citeproc{ref-Jucker2017}{Jucker et al. 2017}). Our analysis showed that
accurately predicting DBH requires both tree height and crown radius
(\citeproc{ref-Jucker2017}{Jucker et al. 2017}), which is especially
important for distinguishing among trees of similar height but varying
DBH (\citeproc{ref-King2005}{King 2005}). Among the tested models, the
\(CR,H\) model
(\(\mathrm{DBH} = a_1 \mathrm{CR}^{b_1} \times \mathrm{H}^{c_1}\))
provided the highest predictive accuracy, a result that contrasts with
the previous study (\citeproc{ref-Jucker2017}{Jucker et al. 2017}),
which concluded that the \(CR \times H\) model
(\(\mathrm{DBH} = a_2(\mathrm{CR} \times \mathrm{H})^{b_2}\)) was the
best for predicting DBH. The \(CR,H\) model significantly improves
predictive accuracy by accounting for the distinct contributions of
crown radius and tree height as independent factors in DBH estimation.
The implications of these differences for biomass estimation are
explored in detail in a later section, where we assess how DBH
prediction accuracy affects allometric biomass calculations.

\subsection{Wood density and tree architecture
diversity}\label{wood-density-and-tree-architecture-diversity}

For the H-DBH relationship, wood density does not strongly affect the
initial growth rate of angiosperm trees but plays a critical role in
determining maximum height and the growth trajectory to achieve that
height, with lighter wood species tending to grow taller and attain
their mature height more rapidly than those with denser wood
(Fig.~\ref{fig-wood1} a-c; Fig.~\ref{fig-wood2} a). In contrast,
gymnosperms exhibit a height growth strategy that is largely independent
of wood density (Fig.~\ref{fig-wood1} d-f). These results are consistent
with the previous study (\citeproc{ref-Poorter2012a}{Poorter et al.
2012}), which showed that among seven angiosperm species, those with
lighter wood exhibited taller maximum heights, while no significant
correlations were observed for six gymnosperm species. A recent study
using the Tallo database alongside two additional datasets reported a
nonsignificant relationship between tree height and wood density
(\citeproc{ref-Jucker2024}{Jucker et al. 2024}). Their analysis included
functional groups (angiosperms vs.~gymnosperms) as a fixed effect,
providing valuable insights into group-level patterns. However, it did
not incorporate interactions between wood density and functional groups,
and thus, the effects of wood density were modeled as uniform across
angiosperms and gymnosperms. In contrast, our approach explicitly
modeled these differences, enabling us to capture functional
group-specific relationships and provide a more nuanced understanding of
height allometry.

In the CR-DBH allometry, our results indicate that both angiosperms and
gymnosperms with denser wood tend to have wider crowns across all trunk
sizes (Fig.~\ref{fig-wood1} g-k). Although the effect of wood density on
CR-DBH allometry is significant, the crown expansion patterns do not
clearly separate light-wood from dense-wood species
(Fig.~\ref{fig-wood2} b, c). This may suggest that crown radius is more
plastic and responsive to local environmental conditions (e.g., light
availability, stand density) than tree height, potentially obscuring the
relationship between wood density and crown radius. Nevertheless, our
findings are consistent with previous work in temperate forests
(\citeproc{ref-Aiba2009}{Aiba and Nakashizuka 2009}), tropical forests
(\citeproc{ref-Iida2012}{Iida et al. 2012}), subtropical forests
(\citeproc{ref-Yang2023}{Yang and Swenson 2023}), and on a global scale
(\citeproc{ref-Jucker2024}{Jucker et al. 2024}), all of which similarly
report that denser wood promotes more efficient horizontal crown
expansion.

In the analysis of DBH allometry, no significant correlation was
observed between wood density and DBH in either angiosperms or
gymnosperms (Table 2). This outcome aligns with observations from
saplings in an Australian subtropical forest
(\citeproc{ref-Kooyman2009}{Kooyman and Westoby 2009}) but contradicts
the expectation that species with lower wood density develop thicker
stems to ensure structural stability, as suggested by prior studies
(\citeproc{ref-King2006}{King et al. 2006};
\citeproc{ref-Anten2009}{Anten and Schieving 2009};
\citeproc{ref-Iida2012}{Iida et al. 2012}). Although wood density is
often regarded as a key functional trait (e.g.,
\citeproc{ref-Anten2009}{Anten and Schieving 2009};
\citeproc{ref-Jucker2024}{Jucker et al. 2024}), variations in tree
allometry across individuals are likely influenced by a combination of
species identity and environmental factors. Furthermore, substantial
measurement errors in tree height and crown radius relative to DBH may
introduce noise that obscures potential relationships between wood
density and stem diameter. These highlight the importance of adopting a
more integrative framework that accounts for additional functional
traits and environmental variables to deepen our understanding of the
factors driving DBH growth.

Our analysis indicated distinct patterns in which wood density has a
greater influence on the interspecific variation in angiosperms than
gymnosperms. Variation in wood density has consequential implications
for the mechanical properties of wood and hence influences the
mechanical stability of stems and branches
(\citeproc{ref-Sterck1998}{Sterck and Bongers 1998}), ultimately
determining the architecture of trees (e.g.,
\citeproc{ref-Poorter2006}{Poorter et al. 2006}). Angiosperms have
denser wood and a wider wood density range than gymnosperms (see
Supplementary Table S2), resulting in greater and more flexible
mechanical support (e.g., \citeproc{ref-Gelder2006}{Gelder et al. 2006};
\citeproc{ref-Swenson2007}{Swenson and Enquist 2007};
\citeproc{ref-Mo2024}{Mo et al. 2024}). This allows them to develop a
more diverse height and crown to optimize light capture, particularly in
environments where light competition is intense, such as tropical
regions. In contrast, gymnosperms predominantly live and often dominate
in extreme environments such as boreal and high-elevation regions
(\citeproc{ref-Swenson2007}{Swenson and Enquist 2007};
\citeproc{ref-Mo2024}{Mo et al. 2024}), prioritising survival and
longevity (\citeproc{ref-Fossdal2024}{Fossdal et al. 2024}). Due to
their simpler and more homogeneous wood anatomy, these species display
lighter wood and a narrower range of wood density (e.g.,
\citeproc{ref-Mo2024}{Mo et al. 2024}), which may result in less
flexibility in the crown architecture. These differences underscore how
the contrasting evolutionary strategies of angiosperms and gymnosperms
shape their architectural diversity, reflecting adaptations to distinct
ecological niches.

\subsection{Consequences for AGB
estimates}\label{consequences-for-agb-estimates}

Our findings revealed minimal differences in AGB estimates derived from
\(\text{H}_\text{WB}\) and \(\text{H}_\text{PL}\) (Fig.~\ref{fig-agb} a,
b). This result can stem from the fact that over 97\% of the trees used
to estimate AGB are small to medium-sized (DBH \textless{} 70 cm
(\citeproc{ref-Slik2013}{Slik et al. 2013})), whose H-DBH relationships
can still be well captured by the power-law function
(Fig.~\ref{fig-compare} a) (e.g., \citeproc{ref-Molto2014}{Molto et al.
2014}). In contrast, individuals with DBH ≥ 70 cm, while typically
forming a small proportion of the tree population, contribute
significantly to biomass storage (e.g., \citeproc{ref-Fauset2015}{Fauset
et al. 2015}). For instance, Slik et al. (\citeproc{ref-Slik2013}{2013})
reported that large trees (DBH ≥ 70 cm) comprised only 2-4\% of the
dataset in tropical forests, yet stored 25-45\% of the total AGB on
average. However, the fundamental assumption of the power-law function
does not fully reflect the asymptotic nature of height-diameter
allometries in large trees (Fig.~\ref{fig-compare} a), potentially
leading to significant inaccuracies in stand-level biomass estimates
(e.g., \citeproc{ref-Feldpausch2011}{Feldpausch et al. 2011};
\citeproc{ref-Fayolle2016}{Fayolle et al. 2016};
\citeproc{ref-Sullivan2018}{Sullivan et al. 2018};
\citeproc{ref-MartinezCano2019}{Martínez Cano et al. 2019}). Given the
disproportionate contribution of large trees in biomass dynamics (e.g.,
\citeproc{ref-Slik2013}{Slik et al. 2013};
\citeproc{ref-Fauset2015}{Fauset et al. 2015};
\citeproc{ref-Ligot2018}{Ligot et al. 2018}), when heights are not
directly measured, height estimates should be derived by fitting
asymptotic models (e.g., Eq.~\ref{eq-gMM}; Eq.~\ref{eq-wb}) to datasets
with sufficient representation of large trees, ensuring the accurate
capture of the saturating component
(\citeproc{ref-Sullivan2018}{Sullivan et al. 2018}).

In the scenarios where DBH is not directly measured, models estimating
AGB using estimated DBH consistently perform poorly (Fig.~\ref{fig-agb}
c, d, e, f). Several reasons may account for this underperformance.
Since the gMM and Weibull models (Eq.~\ref{eq-gMM}; Eq.~\ref{eq-wb}) do
not allow simply switching the roles of explanatory and response
variables. Consequently, we relied solely on the power-law function to
estimate DBH from tree height and crown radius, which may oversimplify
the complex interactions between tree-size variables, potentially
leading to inaccurate DBH estimates (e.g.,
\citeproc{ref-Filipescu2012}{Filipescu et al. 2012}). Additionally, most
existing AGB estimation equations typically include DBH as a squared
term in equations (e.g., \citeproc{ref-Brown1989}{Brown et al. 1989};
\citeproc{ref-Chave2014}{Chave et al. 2014}), meaning that even minor
errors in DBH predictions can significantly affect the accuracy of AGB
estimates (\citeproc{ref-Jucker2017}{Jucker et al. 2017}).
Interestingly, our results revealed that the \(CR,H\) model was more
effective in predicting DBH; however, when these predicted DBH values
were used to estimate AGB, the \(CR \times H\) model exhibited
substantially lower RMSE and bias (Fig.~\ref{fig-agb} c, d). This is
likely due to the \(CR \times H\) model's ability to capture the
inherent interdependence between the crown radius and tree height
(\emph{r} = 0.703), offering a more biologically realistic framework
(e.g., \citeproc{ref-Dormann2013}{Dormann et al. 2013}). The observed
importance of crown radius in improving AGB estimation aligns with prior
studies (e.g., \citeproc{ref-Henry2010}{Henry et al. 2010};
\citeproc{ref-Goodman2014}{Goodman et al. 2014}), whereas our finding of
strong relationships between AGB and crown dimensions is consistent with
those reported by Jucker et al. (\citeproc{ref-Jucker2017}{2017}). While
the interdependence of tree height and crown radius requires careful
model refinement to address potential collinearity issues
(\citeproc{ref-Dormann2013}{Dormann et al. 2013}), their strong
relationship with AGB suggests the feasibility of directly estimating
AGB from EO data, bypassing DBH prediction and simplifying the
estimation process (\citeproc{ref-Jucker2017}{Jucker et al. 2017}).

\section{Conclusions}\label{conclusions}

Our results underscore that saturating functions---particularly the
Weibull model---offer greater accuracy for H--DBH relationships in both
angiosperms and gymnosperms than do conventional power-law models, which
can underestimate biomass in larger trees. For CR--DBH allometry,
different functional forms emerged as optimal for angiosperms
(power-law) and gymnosperms (gMM), reflecting distinct growth
strategies. Although including wood density did not significantly
improve predictive performance, it offered insights into interspecific
trade-offs in tree architecture. These findings are especially important
for large-scale carbon assessments, where even small misjudgments in
tree size can lead to major errors in biomass calculations. Our results
also highlight the importance of incorporating saturating models when
forecasting ecosystem services in forests with a broad range of species
and structural heterogeneity. Future studies should investigate
additional functional traits and environmental factors to improve both
ecological understanding and carbon monitoring practices at global
scales monitoring practices at global scales.

\newpage

\section*{Acknowledgments}\label{acknowledgments}
\addcontentsline{toc}{section}{Acknowledgments}

We sincerely thank the Alliance of International Science Organizations
(ANSO) for providing financial support for TDN's Master's studies.

\section*{Funding}\label{funding}
\addcontentsline{toc}{section}{Funding}

MK was supported by the Xishuangbanna State Rainforest Talent Support
Program (E4BN041B01) and the CAS President's International Fellowship
Initiative (2020FYB0003).

\newpage

\section*{References}\label{references}
\addcontentsline{toc}{section}{References}

\phantomsection\label{refs}
\begin{CSLReferences}{1}{1}
\bibitem[\citeproctext]{ref-Aiba2009}
Aiba M, Nakashizuka T (2009) Architectural differences associated with
adult stature and wood density in 30 temperate tree species. Functional
Ecology 23:265--273.
\url{https://doi.org/10.1111/j.1365-2435.2008.01500.x}

\bibitem[\citeproctext]{ref-Anten2009}
Anten N, Schieving F (2009) The {Role} of {Wood} {Mass} {Density} and
{Mechanical} {Constraints} in the {Economy} of {Tree} {Architecture}.
The American naturalist 175:250--60.
\url{https://doi.org/10.1086/649581}

\bibitem[\citeproctext]{ref-Bazrgar2024}
Bazrgar AB, Thevathasan N, Gordon A, Simpson J (2024) Allometric
equations for estimating aboveground biomass carbon in five tree species
grown in an intercropping agroforestry system in southern {Ontario},
{Canada}. Agroforestry Systems 98:739--749.
\url{https://doi.org/10.1007/s10457-023-00942-z}

\bibitem[\citeproctext]{ref-Brown2004}
Brown JH, Gillooly JF, Allen AP, et al (2004) Toward a {Metabolic}
{Theory} of {Ecology}. Ecology 85:1771--1789.
\url{https://doi.org/10.1890/03-9000}

\bibitem[\citeproctext]{ref-Brown1989}
Brown S, Gillespie AJR, Lugo AE (1989) Biomass {Estimation} {Methods}
for {Tropical} {Forests} with {Applications} to {Forest} {Inventory}
{Data}. Forest Science 35:881--902.
\url{https://doi.org/10.1093/forestscience/35.4.881}

\bibitem[\citeproctext]{ref-Carpenter2017}
Carpenter B, Gelman A, Hoffman MD, et al (2017) Stan: {A}
{Probabilistic} {Programming} {Language}. Journal of Statistical
Software 76:1--32. \url{https://doi.org/10.18637/jss.v076.i01}

\bibitem[\citeproctext]{ref-Chave2014}
Chave J, Réjou-Méchain M, Búrquez A, et al (2014) Improved allometric
models to estimate the aboveground biomass of tropical trees. Global
Change Biology 20:3177--3190. \url{https://doi.org/10.1111/gcb.12629}

\bibitem[\citeproctext]{ref-Dormann2013}
Dormann CF, Elith J, Bacher S, et al (2013) Collinearity: A review of
methods to deal with it and a simulation study evaluating their
performance. Ecography 36:27--46.
\url{https://doi.org/10.1111/j.1600-0587.2012.07348.x}

\bibitem[\citeproctext]{ref-Falster2017}
Falster DS, Brännström Å, Westoby M, Dieckmann U (2017) Multitrait
successional forest dynamics enable diverse competitive coexistence.
Proceedings of the National Academy of Sciences 114:E2719--E2728.
\url{https://doi.org/10.1073/pnas.1610206114}

\bibitem[\citeproctext]{ref-Falster2003}
Falster DS, Westoby M (2003) Plant height and evolutionary games. Trends
in Ecology \& Evolution 18:337--343.
\url{https://doi.org/10.1016/S0169-5347(03)00061-2}

\bibitem[\citeproctext]{ref-Fauset2015}
Fauset S, Johnson MO, Gloor M, et al (2015) Hyperdominance in
{Amazonian} forest carbon cycling. Nature Communications 6:6857.
\url{https://doi.org/10.1038/ncomms7857}

\bibitem[\citeproctext]{ref-Fayolle2016}
Fayolle A, Loubota Panzou GJ, Drouet T, et al (2016) Taller trees,
denser stands and greater biomass in semi-deciduous than in evergreen
lowland central {African} forests. Forest Ecology and Management
374:42--50. \url{https://doi.org/10.1016/j.foreco.2016.04.033}

\bibitem[\citeproctext]{ref-Feldpausch2011}
Feldpausch TR, Banin L, Phillips OL, et al (2011) Height-diameter
allometry of tropical forest trees. Biogeosciences 8:1081--1106.
\url{https://doi.org/10.5194/bg-8-1081-2011}

\bibitem[\citeproctext]{ref-Filipescu2012}
Filipescu CN, Groot A, MacIsaac DA, et al (2012) Prediction of
{Diameter} {Using} {Height} and {Crown} {Attributes}: {A} {Case}
{Study}. Western Journal of Applied Forestry 27:30--35.
\url{https://doi.org/10.1093/wjaf/27.1.30}

\bibitem[\citeproctext]{ref-Fossdal2024}
Fossdal CG, Krokene P, Olsen JE, et al (2024) Epigenetic stress memory
in gymnosperms. Plant Physiology 195:1117--1133.
\url{https://doi.org/10.1093/plphys/kiae051}

\bibitem[\citeproctext]{ref-Gelder2006}
Gelder HAV, Poorter L, Sterck FJ (2006) Wood mechanics, allometry, and
life‐history variation in a tropical rain forest tree community.
https://doi.org/\url{https://doi.org/10.1111/j.1469-8137.2006.01757.x}

\bibitem[\citeproctext]{ref-Gelman2013}
Gelman A, Carlin JB, Stern HS, et al (2015)
\href{https://doi.org/10.1201/b16018}{Bayesian {Data} {Analysis}}, 3rd
edn. Chapman; Hall/CRC, New York

\bibitem[\citeproctext]{ref-Gelman2006}
Gelman A, Hill J (2006) Data analysis using regression and
multilevel/hierarchical models. Cambridge University Press

\bibitem[\citeproctext]{ref-Gelman2008}
Gelman A, Jakulin A, Pittau MG, Su Y-S (2008) A weakly informative
default prior distribution for logistic and other regression models. The
Annals of Applied Statistics 2:1360--1383.
\url{https://doi.org/10.1214/08-AOAS191}

\bibitem[\citeproctext]{ref-Goodman2014}
Goodman RC, Phillips OL, Baker TR (2014) The importance of crown
dimensions to improve tropical tree biomass estimates. Ecological
Applications 24:680--698. \url{https://doi.org/10.1890/13-0070.1}

\bibitem[\citeproctext]{ref-Henry2010}
Henry M, Besnard A, Asante WA, et al (2010) Wood density, phytomass
variations within and among trees, and allometric equations in a
tropical rainforest of {Africa}. Forest Ecology and Management
260:1375--1388. \url{https://doi.org/10.1016/j.foreco.2010.07.040}

\bibitem[\citeproctext]{ref-Hulshof2015}
Hulshof CM, Swenson NG, Weiser MD (2015) Tree height--diameter allometry
across the {United} {States}. Ecology and Evolution 5:1193--1204.
\url{https://doi.org/10.1002/ece3.1328}

\bibitem[\citeproctext]{ref-Iida2012}
Iida Y, Poorter L, Sterck FJ, et al (2012) Wood density explains
architectural differentiation across 145 co-occurring tropical tree
species. Functional Ecology 26:274--282.
\url{https://doi.org/10.1111/j.1365-2435.2011.01921.x}

\bibitem[\citeproctext]{ref-Jucker2017}
Jucker T, Caspersen J, Chave J, et al (2017) Allometric equations for
integrating remote sensing imagery into forest monitoring programmes.
Global Change Biology 23:177--190.
\url{https://doi.org/10.1111/gcb.13388}

\bibitem[\citeproctext]{ref-Jucker2022}
Jucker T, Fischer FJ, Chave J, et al (2022) Tallo: {A} global tree
allometry and crown architecture database. Global Change Biology
28:5254--5268. \url{https://doi.org/10.1111/gcb.16302}

\bibitem[\citeproctext]{ref-Jucker2024}
Jucker T, Fischer FJ, Chave J, et al (2024)
\href{https://doi.org/10.1101/2024.09.14.613032}{The global spectrum of
tree crown architecture}

\bibitem[\citeproctext]{ref-Kattge2020}
Kattge J, Bönisch G, Díaz S, et al (2020) {TRY} plant trait database --
enhanced coverage and open access. Global Change Biology 26:119--188.
\url{https://doi.org/10.1111/gcb.14904}

\bibitem[\citeproctext]{ref-King2005}
King DA (2005) Linking tree form, allocation and growth with an
allometrically explicit model. Ecological Modelling 185:77--91.
\url{https://doi.org/10.1016/j.ecolmodel.2004.11.017}

\bibitem[\citeproctext]{ref-King2006}
King DA, Davies SJ, Tan S, Noor NSMd (2006) The role of wood density and
stem support costs in the growth and mortality of tropical trees.
Journal of Ecology 94:670--680.
\url{https://doi.org/10.1111/j.1365-2745.2006.01112.x}

\bibitem[\citeproctext]{ref-Kooyman2009}
Kooyman RM, Westoby M (2009) Costs of height gain in rainforest
saplings: Main-stem scaling, functional traits and strategy variation
across 75 species. Annals of Botany 104:987--993.
\url{https://doi.org/10.1093/aob/mcp185}

\bibitem[\citeproctext]{ref-Landau2021}
Landau WM (2021) \href{https://doi.org/10.21105/joss.02959}{The targets
r package: A dynamic make-like function-oriented pipeline toolkit for
reproducibility and high-performance computing}. Journal of Open Source
Software 6:2959

\bibitem[\citeproctext]{ref-Laurans2024}
Laurans M, Munoz F, Charles-Dominique T, et al (2024) Why incorporate
plant architecture into trait-based ecology? Trends in Ecology \&
Evolution. \url{https://doi.org/10.1016/j.tree.2023.11.011}

\bibitem[\citeproctext]{ref-Ligot2018}
Ligot G, Gourlet-Fleury S, Ouédraogo D-Y, et al (2018) The limited
contribution of large trees to annual biomass production in an
old-growth tropical forest. Ecological Applications: A Publication of
the Ecological Society of America 28:1273--1281.
\url{https://doi.org/10.1002/eap.1726}

\bibitem[\citeproctext]{ref-Lines2012}
Lines ER, Zavala MA, Purves DW, Coomes DA (2012) Predictable changes in
aboveground allometry of trees along gradients of temperature, aridity
and competition. Global Ecology and Biogeography 21:1017--1028.
\url{https://doi.org/10.1111/j.1466-8238.2011.00746.x}

\bibitem[\citeproctext]{ref-Loehle2016}
Loehle C (2016) Biomechanical constraints on tree architecture. Trees
30:2061--2070. \url{https://doi.org/10.1007/s00468-016-1433-2}

\bibitem[\citeproctext]{ref-LoubotaPanzou2018}
Loubota Panzou GJ, Ligot G, Gourlet-Fleury S, et al (2018) Architectural
differences associated with functional traits among 45 coexisting tree
species in {Central} {Africa}. Functional Ecology 32:2583--2593.
\url{https://doi.org/10.1111/1365-2435.13198}

\bibitem[\citeproctext]{ref-MartinezCano2019}
Martínez Cano I, Muller-Landau HC, Wright SJ, et al (2019) Tropical tree
height and crown allometries for the {Barro} {Colorado} {Nature}
{Monument}, {Panama}: A comparison of alternative hierarchical models
incorporating interspecific variation in relation to life history
traits. Biogeosciences 16:847--862.
\url{https://doi.org/10.5194/bg-16-847-2019}

\bibitem[\citeproctext]{ref-Mikola1938}
Mikola P (1938) \href{https://www.silvafennica.fi/article/4546}{Crown
and stem form of {Norway} spruce in the snow damage areas of {Maanselkä}
in {Northern} {Finland}}. Silva Fennica

\bibitem[\citeproctext]{ref-Mo2024}
Mo L, Crowther TW, Maynard DS, et al (2024) The global distribution and
drivers of wood density and their impact on forest carbon stocks. Nature
Ecology \& Evolution 1--18.
\url{https://doi.org/10.1038/s41559-024-02564-9}

\bibitem[\citeproctext]{ref-Molto2014}
Molto Q, Hérault B, Boreux J-J, et al (2014) Predicting tree heights for
biomass estimates in tropical forests \&ndash; a test from {French}
{Guiana}. Biogeosciences 11:3121--3130.
\url{https://doi.org/10.5194/bg-11-3121-2014}

\bibitem[\citeproctext]{ref-Moore2010}
Moore JR (2010) Allometric equations to predict the total above-ground
biomass of radiata pine trees. Annals of Forest Science 67:806--806.
\url{https://doi.org/10.1051/forest/2010042}

\bibitem[\citeproctext]{ref-Niklas2007}
Niklas KJ (2007) Maximum plant height and the biophysical factors that
limit it. Tree Physiology 27:433--440.
\url{https://doi.org/10.1093/treephys/27.3.433}

\bibitem[\citeproctext]{ref-Niklas1994}
Niklas KJ (1994) Plant allometry: The scaling of form and process.
University of Chicago Press, Chicago

\bibitem[\citeproctext]{ref-Poorter2012a}
Poorter H, Niklas KJ, Reich PB, et al (2012) Biomass allocation to
leaves, stems and roots: Meta-analyses of interspecific variation and
environmental control. New Phytologist 193:30--50.
\url{https://doi.org/10.1111/j.1469-8137.2011.03952.x}

\bibitem[\citeproctext]{ref-Poorter2006}
Poorter L, Bongers L, Bongers F (2006) Architecture of 54
{Moist}-{Forest} {Tree} {Species}: {Traits}, {Trade}-{Offs}, and
{Functional} {Groups}. Ecology 87:1289--1301.
\url{https://doi.org/10.1890/0012-9658(2006)87\%5B1289:AOMTST\%5D2.0.CO;2}

\bibitem[\citeproctext]{ref-RCoreTeam2023}
R Core Team (2023) \href{https://www.R-project.org/}{R: A language and
environment for statistical computing}. R Foundation for Statistical
Computing, Vienna, Austria

\bibitem[\citeproctext]{ref-Shendryk2016}
Shendryk I, Broich M, Tulbure MG, Alexandrov SV (2016) Bottom-up
delineation of individual trees from full-waveform airborne laser scans
in a structurally complex eucalypt forest. Remote Sensing of Environment
173:69--83. \url{https://doi.org/10.1016/j.rse.2015.11.008}

\bibitem[\citeproctext]{ref-Shinozaki1964a}
Shinozaki K, Yoda K, Hozumi K, Kira T (1964) A quantitative analysis of
plant form-the pipe model theory: I. Basic analyses. Japanese Journal of
ecology 14:97--105

\bibitem[\citeproctext]{ref-Sileshi2023}
Sileshi GW, Nath AJ, Kuyah S (2023) Allometric scaling and allocation
patterns: {Implications} for predicting productivity across plant
communities. Frontiers in Forests and Global Change 5:
\url{https://doi.org/10.3389/ffgc.2022.1084480}

\bibitem[\citeproctext]{ref-Slik2013}
Slik JWF, Paoli G, McGuire K, et al (2013) Large trees drive forest
aboveground biomass variation in moist lowland forests across the
tropics. Global Ecology and Biogeography 22:1261--1271.
\url{https://doi.org/10.1111/geb.12092}

\bibitem[\citeproctext]{ref-Song2024}
Song X, Li J, Zeng X (2024) Parameterization of height--diameter and
crown radius--diameter relationships across the globe. Journal of Plant
Ecology 17:rtae005. \url{https://doi.org/10.1093/jpe/rtae005}

\bibitem[\citeproctext]{ref-Sterck1998}
Sterck F, Bongers F (1998)
\href{https://www.ncbi.nlm.nih.gov/pubmed/21684910}{Ontogenetic changes
in size, allometry, and mechanical design of tropical rain forest
trees}. American Journal of Botany 85:266

\bibitem[\citeproctext]{ref-Sullivan2018}
Sullivan MJP, Lewis SL, Hubau W, et al (2018) Field methods for sampling
tree height for tropical forest biomass estimation. Methods in Ecology
and Evolution 9:1179--1189.
\url{https://doi.org/10.1111/2041-210X.12962}

\bibitem[\citeproctext]{ref-Swenson2007}
Swenson N, Enquist B (2007) Ecological and evolutionary determinants of
a key plant functional trait: {Wood} density and its community-wide
variation across latitude and elevation. American journal of botany
94:451--9. \url{https://doi.org/10.3732/ajb.94.3.451}

\bibitem[\citeproctext]{ref-Tatsumi2023}
Tatsumi S, Yamaguchi K, Furuya N (2023) {ForestScanner}: {A} mobile
application for measuring and mapping trees with {LiDAR}-equipped
{iPhone} and {iPad}. Methods in Ecology and Evolution 14:1603--1609.
\url{https://doi.org/10.1111/2041-210X.13900}

\bibitem[\citeproctext]{ref-Thomas1996}
Thomas SC (1996) Asymptotic height as a predictor of growth and
allometric characteristics in malaysian rain forest trees. American
Journal of Botany 83:556--566.
\url{https://doi.org/10.1002/j.1537-2197.1996.tb12739.x}

\bibitem[\citeproctext]{ref-Vehtari2017}
Vehtari A, Gelman A, Gabry J (2017) Practical {Bayesian} model
evaluation using leave-one-out cross-validation and {WAIC}. Statistics
and Computing 27:1413--1432.
\url{https://doi.org/10.1007/s11222-016-9696-4}

\bibitem[\citeproctext]{ref-Vehtari2021}
Vehtari A, Gelman A, Simpson D, et al (2021) Rank-{Normalization},
{Folding}, and {Localization}: {An} {Improved} {Rˆ} for {Assessing}
{Convergence} of {MCMC} (with {Discussion}). Bayesian Analysis
16:667--718. \url{https://doi.org/10.1214/20-BA1221}

\bibitem[\citeproctext]{ref-West1999}
West G, Brown J, Enquist B (1999) A general model for the structure and
allometry of plant vascular systems. Nature 400:664--667.
\url{https://doi.org/10.1038/23251}

\bibitem[\citeproctext]{ref-Yang2023}
Yang J, Swenson NG (2023) Height and crown allometries and their
relationship with functional traits: {An} example from a subtropical wet
forest. Ecology and Evolution 13:e9804.
\url{https://doi.org/10.1002/ece3.9804}

\end{CSLReferences}




\end{document}
