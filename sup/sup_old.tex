% Options for packages loaded elsewhere
\PassOptionsToPackage{unicode}{hyperref}
\PassOptionsToPackage{hyphens}{url}
\PassOptionsToPackage{dvipsnames,svgnames,x11names}{xcolor}
%
\documentclass[
  12pt,
  letterpaper,
  DIV=11,
  numbers=noendperiod]{scrartcl}

\usepackage{amsmath,amssymb}
\usepackage{iftex}
\ifPDFTeX
  \usepackage[T1]{fontenc}
  \usepackage[utf8]{inputenc}
  \usepackage{textcomp} % provide euro and other symbols
\else % if luatex or xetex
  \usepackage{unicode-math}
  \defaultfontfeatures{Scale=MatchLowercase}
  \defaultfontfeatures[\rmfamily]{Ligatures=TeX,Scale=1}
\fi
\usepackage{lmodern}
\ifPDFTeX\else  
    % xetex/luatex font selection
\fi
% Use upquote if available, for straight quotes in verbatim environments
\IfFileExists{upquote.sty}{\usepackage{upquote}}{}
\IfFileExists{microtype.sty}{% use microtype if available
  \usepackage[]{microtype}
  \UseMicrotypeSet[protrusion]{basicmath} % disable protrusion for tt fonts
}{}
\makeatletter
\@ifundefined{KOMAClassName}{% if non-KOMA class
  \IfFileExists{parskip.sty}{%
    \usepackage{parskip}
  }{% else
    \setlength{\parindent}{0pt}
    \setlength{\parskip}{6pt plus 2pt minus 1pt}}
}{% if KOMA class
  \KOMAoptions{parskip=half}}
\makeatother
\usepackage{xcolor}
\usepackage[margin=1in]{geometry}
\setlength{\emergencystretch}{3em} % prevent overfull lines
\setcounter{secnumdepth}{-\maxdimen} % remove section numbering
% Make \paragraph and \subparagraph free-standing
\makeatletter
\ifx\paragraph\undefined\else
  \let\oldparagraph\paragraph
  \renewcommand{\paragraph}{
    \@ifstar
      \xxxParagraphStar
      \xxxParagraphNoStar
  }
  \newcommand{\xxxParagraphStar}[1]{\oldparagraph*{#1}\mbox{}}
  \newcommand{\xxxParagraphNoStar}[1]{\oldparagraph{#1}\mbox{}}
\fi
\ifx\subparagraph\undefined\else
  \let\oldsubparagraph\subparagraph
  \renewcommand{\subparagraph}{
    \@ifstar
      \xxxSubParagraphStar
      \xxxSubParagraphNoStar
  }
  \newcommand{\xxxSubParagraphStar}[1]{\oldsubparagraph*{#1}\mbox{}}
  \newcommand{\xxxSubParagraphNoStar}[1]{\oldsubparagraph{#1}\mbox{}}
\fi
\makeatother

\usepackage{color}
\usepackage{fancyvrb}
\newcommand{\VerbBar}{|}
\newcommand{\VERB}{\Verb[commandchars=\\\{\}]}
\DefineVerbatimEnvironment{Highlighting}{Verbatim}{commandchars=\\\{\}}
% Add ',fontsize=\small' for more characters per line
\usepackage{framed}
\definecolor{shadecolor}{RGB}{241,243,245}
\newenvironment{Shaded}{\begin{snugshade}}{\end{snugshade}}
\newcommand{\AlertTok}[1]{\textcolor[rgb]{0.68,0.00,0.00}{#1}}
\newcommand{\AnnotationTok}[1]{\textcolor[rgb]{0.37,0.37,0.37}{#1}}
\newcommand{\AttributeTok}[1]{\textcolor[rgb]{0.40,0.45,0.13}{#1}}
\newcommand{\BaseNTok}[1]{\textcolor[rgb]{0.68,0.00,0.00}{#1}}
\newcommand{\BuiltInTok}[1]{\textcolor[rgb]{0.00,0.23,0.31}{#1}}
\newcommand{\CharTok}[1]{\textcolor[rgb]{0.13,0.47,0.30}{#1}}
\newcommand{\CommentTok}[1]{\textcolor[rgb]{0.37,0.37,0.37}{#1}}
\newcommand{\CommentVarTok}[1]{\textcolor[rgb]{0.37,0.37,0.37}{\textit{#1}}}
\newcommand{\ConstantTok}[1]{\textcolor[rgb]{0.56,0.35,0.01}{#1}}
\newcommand{\ControlFlowTok}[1]{\textcolor[rgb]{0.00,0.23,0.31}{\textbf{#1}}}
\newcommand{\DataTypeTok}[1]{\textcolor[rgb]{0.68,0.00,0.00}{#1}}
\newcommand{\DecValTok}[1]{\textcolor[rgb]{0.68,0.00,0.00}{#1}}
\newcommand{\DocumentationTok}[1]{\textcolor[rgb]{0.37,0.37,0.37}{\textit{#1}}}
\newcommand{\ErrorTok}[1]{\textcolor[rgb]{0.68,0.00,0.00}{#1}}
\newcommand{\ExtensionTok}[1]{\textcolor[rgb]{0.00,0.23,0.31}{#1}}
\newcommand{\FloatTok}[1]{\textcolor[rgb]{0.68,0.00,0.00}{#1}}
\newcommand{\FunctionTok}[1]{\textcolor[rgb]{0.28,0.35,0.67}{#1}}
\newcommand{\ImportTok}[1]{\textcolor[rgb]{0.00,0.46,0.62}{#1}}
\newcommand{\InformationTok}[1]{\textcolor[rgb]{0.37,0.37,0.37}{#1}}
\newcommand{\KeywordTok}[1]{\textcolor[rgb]{0.00,0.23,0.31}{\textbf{#1}}}
\newcommand{\NormalTok}[1]{\textcolor[rgb]{0.00,0.23,0.31}{#1}}
\newcommand{\OperatorTok}[1]{\textcolor[rgb]{0.37,0.37,0.37}{#1}}
\newcommand{\OtherTok}[1]{\textcolor[rgb]{0.00,0.23,0.31}{#1}}
\newcommand{\PreprocessorTok}[1]{\textcolor[rgb]{0.68,0.00,0.00}{#1}}
\newcommand{\RegionMarkerTok}[1]{\textcolor[rgb]{0.00,0.23,0.31}{#1}}
\newcommand{\SpecialCharTok}[1]{\textcolor[rgb]{0.37,0.37,0.37}{#1}}
\newcommand{\SpecialStringTok}[1]{\textcolor[rgb]{0.13,0.47,0.30}{#1}}
\newcommand{\StringTok}[1]{\textcolor[rgb]{0.13,0.47,0.30}{#1}}
\newcommand{\VariableTok}[1]{\textcolor[rgb]{0.07,0.07,0.07}{#1}}
\newcommand{\VerbatimStringTok}[1]{\textcolor[rgb]{0.13,0.47,0.30}{#1}}
\newcommand{\WarningTok}[1]{\textcolor[rgb]{0.37,0.37,0.37}{\textit{#1}}}

\providecommand{\tightlist}{%
  \setlength{\itemsep}{0pt}\setlength{\parskip}{0pt}}\usepackage{longtable,booktabs,array}
\usepackage{calc} % for calculating minipage widths
% Correct order of tables after \paragraph or \subparagraph
\usepackage{etoolbox}
\makeatletter
\patchcmd\longtable{\par}{\if@noskipsec\mbox{}\fi\par}{}{}
\makeatother
% Allow footnotes in longtable head/foot
\IfFileExists{footnotehyper.sty}{\usepackage{footnotehyper}}{\usepackage{footnote}}
\makesavenoteenv{longtable}
\usepackage{graphicx}
\makeatletter
\newsavebox\pandoc@box
\newcommand*\pandocbounded[1]{% scales image to fit in text height/width
  \sbox\pandoc@box{#1}%
  \Gscale@div\@tempa{\textheight}{\dimexpr\ht\pandoc@box+\dp\pandoc@box\relax}%
  \Gscale@div\@tempb{\linewidth}{\wd\pandoc@box}%
  \ifdim\@tempb\p@<\@tempa\p@\let\@tempa\@tempb\fi% select the smaller of both
  \ifdim\@tempa\p@<\p@\scalebox{\@tempa}{\usebox\pandoc@box}%
  \else\usebox{\pandoc@box}%
  \fi%
}
% Set default figure placement to htbp
\def\fps@figure{htbp}
\makeatother

\usepackage{booktabs}
\usepackage{longtable}
\usepackage{array}
\usepackage{multirow}
\usepackage{wrapfig}
\usepackage{float}
\usepackage{colortbl}
\usepackage{pdflscape}
\usepackage{tabu}
\usepackage{threeparttable}
\usepackage{threeparttablex}
\usepackage[normalem]{ulem}
\usepackage{makecell}
\usepackage{xcolor}
\usepackage[default]{sourcesanspro}
\usepackage{sourcecodepro}
\usepackage{lineno}
\usepackage{setspace}
\doublespacing
\linenumbers
\renewcommand{\thetable}{S\arabic{table}}
\renewcommand{\thefigure}{S\arabic{figure}}
\KOMAoption{captions}{tableheading}
\makeatletter
\@ifpackageloaded{caption}{}{\usepackage{caption}}
\AtBeginDocument{%
\ifdefined\contentsname
  \renewcommand*\contentsname{Table of contents}
\else
  \newcommand\contentsname{Table of contents}
\fi
\ifdefined\listfigurename
  \renewcommand*\listfigurename{List of Figures}
\else
  \newcommand\listfigurename{List of Figures}
\fi
\ifdefined\listtablename
  \renewcommand*\listtablename{List of Tables}
\else
  \newcommand\listtablename{List of Tables}
\fi
\ifdefined\figurename
  \renewcommand*\figurename{Figure}
\else
  \newcommand\figurename{Figure}
\fi
\ifdefined\tablename
  \renewcommand*\tablename{Table}
\else
  \newcommand\tablename{Table}
\fi
}
\@ifpackageloaded{float}{}{\usepackage{float}}
\floatstyle{ruled}
\@ifundefined{c@chapter}{\newfloat{codelisting}{h}{lop}}{\newfloat{codelisting}{h}{lop}[chapter]}
\floatname{codelisting}{Listing}
\newcommand*\listoflistings{\listof{codelisting}{List of Listings}}
\makeatother
\makeatletter
\makeatother
\makeatletter
\@ifpackageloaded{caption}{}{\usepackage{caption}}
\@ifpackageloaded{subcaption}{}{\usepackage{subcaption}}
\makeatother

\usepackage{bookmark}

\IfFileExists{xurl.sty}{\usepackage{xurl}}{} % add URL line breaks if available
\urlstyle{same} % disable monospaced font for URLs
\hypersetup{
  colorlinks=true,
  linkcolor={blue},
  filecolor={Maroon},
  citecolor={Blue},
  urlcolor={Blue},
  pdfcreator={LaTeX via pandoc}}


\author{}
\date{}

\begin{document}


\emph{Supplement of}

\textbf{Saturating allometric relationships reveal how wood density
shapes global tree architecture}

\newpage

This supplement includes the following materials:

\section{Section S1. Extra figures and
tables}\label{section-s1.-extra-figures-and-tables}

\textbf{Table S1.} List of sources for wood density data used in this
study, as released in version 6 of the TRY database. The table includes
authors' names, dataset IDs, and the name of each dataset.

\textbf{Table S2.} Characteristics of sub-datasets were used to select
the best predictive model of each allometric relationship for
angiosperms and gymnosperms.

\textbf{Table S3.} Parameter estimates for all hierarchical models for
tree height allometry.

\textbf{Table S4.} Parameter estimates for all hierarchical models for
crown radius allometry.

\textbf{Table S5.} Parameter estimates for all hierarchical models for
DBH allometry.

\textbf{Table S6.} Parameter estimates for predicting tree height (m)
from DBH (cm) of 1290 species (see file
TableS6\_height\_sp\_estimates.xlsx).

\textbf{Table S7.} Parameter estimates for predicting crown radius (m)
from DBH (cm) of 821 species (see file
TableS7\_crown\_radius\_sp\_estimates.xlsx).

\textbf{Table S8.} Parameter estimates for predicting DBH (cm) from
crown radius (m) and tree height (m) of 800 species (see file
TableS8\_dbh\_sp\_estimates.xlsx).

\textbf{Figure S1.} Function curves are used to describe allometric
relationships of tree dimensions with DBH in log scale.

\section{Section S2. Stan code used to fit alternative allometric
models}\label{section-s2.-stan-code-used-to-fit-alternative-allometric-models}

\newpage

\section{Section S1. Extra tables and
figures}\label{section-s1.-extra-tables-and-figures}

\textbf{Table S1}: List of sources for wood density data utilized in
this study, as released in version 6 of the TRY database. The table
includes authors' names, dataset IDs, and the name of each dataset.

\begingroup\fontsize{10}{12}\selectfont

\begin{longtable*}[t]{ccccc}
\toprule
No. & LastName & FirstName & DatasetID & Dataset\\
\midrule
1 & Cornelissen & Johannes & 37 & Sheffield Database\\
2 & Swaine & Emily & 51 & Tropical Plant Traits From Borneo Database\\
3 & Kleyer & Michael & 25 & The LEDA Traitbase\\
4 & Cornwell & Will & 55 & Jasper Ridge Californian Woody Plants Database\\
5 & Lloyd & Jon & 34 & The RAINFOR Plant Trait Database\\
\bottomrule
\end{longtable*}
\endgroup{}

\newpage

\textbf{Table S2}: Characteristics of sub-datasets used to select the
best predictive model of each allometric relationship for angiosperms
and gymnosperms. The ranges represent the minimum and maximum values for
the respective variables: DBH (cm), tree height (H, m), crown radius
(CR, m), and wood density (WD, g cm\(^{-3}\)).

\begingroup\fontsize{10}{12}\selectfont

\begin{longtable*}[t]{ccccc}
\toprule
Dependent variables & Predictor variable & Characteristics & Angiosperm & Gymnosperm\\
\midrule
H & DBH & Number of trees & 79,332 & 6,306\\
H & DBH & Number of species & 1,214 & 76\\
H & DBH & Dependent variable range & 1.3 - 94 & 1.3 - 115.8\\
H & DBH & Predictor variable range & 1 - 560 & 1 - 648\\
H & DBH & WD range & 0.131 - 1.17 & 0.32 - 0.689\\
\addlinespace
CR & DBH & Number of trees & 45,350 & 4,632\\
CR & DBH & Number of species & 762 & 59\\
CR & DBH & Dependent variable range & 0.05 - 24.25 & 0.1 - 8.75\\
CR & DBH & Predictor variable range & 1 - 385 & 1 - 266\\
CR & DBH & WD range & 0.131 - 1.15 & 0.32 - 0.689\\
\addlinespace
DBH & CR, H & Number of trees & 41,713 & 3,661\\
DBH & CR, H & Number of species & 744 & 56\\
DBH & CR, H & Dependent variable range & 1 - 385 & 1 - 266\\
DBH & CR, H & Predictor variable range & CR: 0.1 - 24.25 H: 1.3 - 70.6 & CR: 0.1 - 10.65 H: 1.3 - 46.2\\
DBH & CR, H & WD range & 0.131 - 1.15 & 0.32 - 0.651\\
\addlinespace
DBH & CR × H & Number of trees & 41,733 & 3,716\\
DBH & CR × H & Number of species & 743 & 56\\
DBH & CR × H & Dependent variable range & 1 - 385 & 1 - 266\\
DBH & CR × H & Predictor variable range & CR: 0.1 - 24.25 H: 1.3 - 70.6 & CR: 0.15 - 10.65 H: 1.5 - 53.9\\
DBH & CR × H & WD range & 0.131 - 1.15 & 0.32 - 0.651\\
\addlinespace
DBH & CR & Number of trees & 41,733 & 3,702\\
DBH & CR & Number of species & 744 & 56\\
DBH & CR & Dependent variable range & 1 - 385 & 1 - 266\\
DBH & CR & Predictor variable range & 0.1 - 24.25 & 0.15 - 8.75\\
DBH & CR & WD range & 0.131 - 1.15 & 0.32 - 0.651\\
\addlinespace
DBH & H & Number of trees & 41,689 & 3,648\\
DBH & H & Number of species & 746 & 55\\
DBH & H & Dependent variable range & 1 - 385 & 1 - 266\\
DBH & H & Predictor variable range & 1.3 - 70.6 & 1.5 - 46.2\\
DBH & H & WD range & 0.131 - 1.15 & 0.32 - 0.651\\
\bottomrule
\end{longtable*}
\endgroup{}

\newpage

\textbf{Table S3}: Posterior estimates of the parameters of the
hierarchical models for tree height allometry. Table entries correspond
to the median and 95\% CIs for the community-level parameters of each
allometric function.

\begingroup\fontsize{10}{12}\selectfont

\begin{longtable*}[t]{ccccccc}
\toprule
Division & Functional\_form & Wood\_density & Parameter & Intercept\_CI & Slope\_CI & Tau\_CI\\
\midrule
Angiosperm & Power-Law & Without & a & 2.55 (2.50, 2.60) & - & 0.31 (0.30, 0.33)\\
Angiosperm & Power-Law & Without & b & 0.54 (0.53, 0.54) & - & 0.11 (0.10, 0.11)\\
Angiosperm & gMM & Without & a & 77.90 (73.81, 82.50) & - & 0.14 (0.11, 0.16)\\
Angiosperm & gMM & Without & b & 0.64 (0.63, 0.65) & - & 0.15 (0.14, 0.16)\\
Angiosperm & gMM & Without & k & 34.95 (33.24, 36.86) & - & 8.95 (8.23, 9.66)\\
\addlinespace
Angiosperm & Weibull & Without & a & 49.99 (48.04, 52.17) & - & 0.11 (0.08, 0.13)\\
Angiosperm & Weibull & Without & b & 0.05 (0.05, 0.05) & - & 0.01 (0.01, 0.01)\\
Angiosperm & Weibull & Without & k & 0.63 (0.62, 0.64) & - & 0.14 (0.14, 0.15)\\
Angiosperm & Weibull & With & a & 50.04 (48.02, 52.28) & 1.00 (0.97, 1.04) & 0.11 (0.09, 0.13)\\
Angiosperm & Weibull & With & b & 0.05 (0.05, 0.05) & -0.000 (-0.002, 0.001) & 0.01 (0.01, 0.01)\\
\addlinespace
Angiosperm & Weibull & With & k & 0.63 (0.62, 0.64) & -0.02 (-0.03, -0.006) & 0.14 (0.13, 0.15)\\
Gymnosperm & Power-Law & Without & a & 1.43 (1.30, 1.56) & - & 0.37 (0.30, 0.45)\\
Gymnosperm & Power-Law & Without & b & 0.70 (0.67, 0.73) & - & 0.11 (0.09, 0.14)\\
Gymnosperm & gMM & Without & a & 2004.65 (386.15, 7618.20) & - & 0.36 (0.30, 0.43)\\
Gymnosperm & gMM & Without & b & 0.72 (0.69, 0.75) & - & 0.11 (0.09, 0.14)\\
\addlinespace
Gymnosperm & gMM & Without & k & 1427 (287.33, 5308.34) & - & 5.90 (0.08, 32.24)\\
Gymnosperm & Weibull & Without & a & 117.41 (89.22, 175.13) & - & 0.09 (0.004, 0.23)\\
Gymnosperm & Weibull & Without & b & 0.01 (0.008, 0.02) & - & 0.004 (0.002, 0.005)\\
Gymnosperm & Weibull & Without & k & 0.75 (0.72, 0.78) & - & 0.11 (0.09, 0.14)\\
Gymnosperm & Weibull & With & a & 114.39 (85.14, 170.40) & 0.97 (0.78, 1.21) & 0.10 (0.005, 0.26)\\
\addlinespace
Gymnosperm & Weibull & With & b & 0.01 (0.008, 0.02) & 0.001 (-0.002, 0.003) & 0.004 (0.002, 0.006)\\
Gymnosperm & Weibull & With & k & 0.75 (0.72, 0.78) & -0.02 (-0.05, 0.008) & 0.11 (0.09, 0.14)\\
\bottomrule
\end{longtable*}
\endgroup{}

\newpage

\textbf{Table S4}: Posterior estimates of the parameters of the
hierarchical models for crown radius allometry. Table entries correspond
to the median and 95\% CIs for the community-level parameters of each
allometric function.

\begingroup\fontsize{10}{12}\selectfont

\begin{longtable*}[t]{ccccccc}
\toprule
Division & Functional\_form & Wood\_density & Parameter & Intercept\_CI & Slope\_CI & Tau\_CI\\
\midrule
Angiosperm & Power-Law & Without & a & 0.35 (0.34, 0.36) & - & 0.34 (0.32, 0.36)\\
Angiosperm & Power-Law & Without & b & 0.63 (0.62, 0.63) & - & 0.09 (0.09, 0.10)\\
Angiosperm & Power-Law & With & a & 0.35 (0.34, 0.36) & 0.07 (0.04, 0.10) & 0.33 (0.31, 0.35)\\
Angiosperm & Power-Law & With & b & 0.63 (0.62, 0.63) & -0.01 (-0.02, -0.003) & 0.09 (0.08, 0.10)\\
Angiosperm & gMM & Without & a & 1382.37 (625.38, 2564.02) & - & 0.34 (0.32, 0.36)\\
\addlinespace
Angiosperm & gMM & Without & b & 0.63 (0.62, 0.64) & - & 0.09 (0.09, 0.10)\\
Angiosperm & gMM & Without & k & 3974.91 (1807.16, 7385.72) & - & 7.24 (0.09, 50.73)\\
Angiosperm & Weibull & Without & a & 458.39 (174.88, 1761.34) & - & 0.31 (0.25, 0.35)\\
Angiosperm & Weibull & Without & b & 0.001 (0.000, 0.002) & - & 9.67e-05 (5.09e-06, 0.000)\\
Angiosperm & Weibull & Without & k & 0.63 (0.62, 0.64) & - & 0.09 (0.09, 0.10)\\
\addlinespace
Gymnosperm & Power-Law & Without & a & 0.30 (0.27, 0.33) & - & 0.32 (0.25, 0.41)\\
Gymnosperm & Power-Law & Without & b & 0.62 (0.59, 0.65) & - & 0.09 (0.06, 0.11)\\
Gymnosperm & gMM & Without & a & 769.41 (227.35, 1818.31) & - & 0.37 (0.28, 0.47)\\
Gymnosperm & gMM & Without & b & 0.63 (0.60, 0.66) & - & 0.10 (0.08, 0.13)\\
Gymnosperm & gMM & Without & k & 2657.08 (808.71, 6274.34) & - & 18.98 (0.11, 200.19)\\
\addlinespace
Gymnosperm & gMM & With & a & 975.00 (311.90, 2010.23) & 0.99 (0.77, 1.27) & 0.37 (0.28, 0.47)\\
Gymnosperm & gMM & With & b & 0.63 (0.60, 0.66) & 0.03 (0.005, 0.06) & 0.09 (0.07, 0.13)\\
Gymnosperm & gMM & With & k & 3371.46 (1094.23, 6977.77) & 35.71 (-768.70, 913.80) & 9.36 (0.10, 52.32)\\
Gymnosperm & Weibull & Without & a & 61.02 (21.51, 321.80) & - & 0.20 (0.01, 0.41)\\
Gymnosperm & Weibull & Without & b & 0.005 (0.001, 0.01) & - & 0.002 (5.17e-05, 0.005)\\
\addlinespace
Gymnosperm & Weibull & Without & k & 0.64 (0.61, 0.67) & - & 0.10 (0.08, 0.13)\\
\bottomrule
\end{longtable*}
\endgroup{}

\newpage

\textbf{Table S5}: Posterior estimates of the parameters of the
hierarchical models for DBH allometry. Table entries correspond to the
median and 95\% CIs for the community-level parameters of each
allometric function.

\begingroup\fontsize{10}{12}\selectfont

\begin{longtable*}[t]{cccccccc}
\toprule
Division & Predictor & Functional\_form & Wood\_density & Parameter & Intercept\_CI & Slope\_CI & Tau\_CI\\
\midrule
Angiosperm & CR, H & Power-Law & Without & a & 0.95 (0.93, 0.97) & - & 0.30 (0.28, 0.32)\\
Angiosperm & CR, H & Power-Law & Without & b & 0.37 (0.36, 0.39) & - & 0.13 (0.13, 0.15)\\
Angiosperm & CR, H & Power-Law & Without & c & 0.59 (0.57, 0.60) & - & 0.17 (0.16, 0.18)\\
Angiosperm & CR, H & Power-Law & With & a & 0.95 (0.93, 0.97) & -0.003 (-0.03, 0.02) & 0.30 (0.28, 0.32)\\
Angiosperm & CR, H & Power-Law & With & b & 0.37 (0.36, 0.39) & -0.003 (-0.01, 0.009) & 0.14 (0.13, 0.15)\\
\addlinespace
Angiosperm & CR, H & Power-Law & With & c & 0.59 (0.57, 0.60) & 0.005 (-0.010, 0.02) & 0.17 (0.16, 0.18)\\
Angiosperm & CR × H & Power-Law & Without & a & 0.96 (0.93, 0.98) & - & 0.31 (0.29, 0.32)\\
Angiosperm & CR × H & Power-Law & Without & b & 0.87 (0.86, 0.89) & - & 0.17 (0.16, 0.18)\\
Angiosperm & CR & Power-Law & Without & a & 0.95 (0.92, 0.98) & - & 0.42 (0.40, 0.45)\\
Angiosperm & CR & Power-Law & Without & b & 0.72 (0.70, 0.74) & - & 0.24 (0.22, 0.25)\\
\addlinespace
Angiosperm & H & Power-Law & Without & a & 0.97 (0.94, 0.99) & - & 0.35 (0.33, 0.37)\\
Angiosperm & H & Power-Law & Without & b & 0.87 (0.85, 0.88) & - & 0.20 (0.19, 0.21)\\
Gymnosperm & CR, H & Power-Law & Without & a & 0.97 (0.90, 1.05) & - & 0.27 (0.22, 0.34)\\
Gymnosperm & CR, H & Power-Law & Without & b & 0.41 (0.37, 0.45) & - & 0.11 (0.09, 0.15)\\
Gymnosperm & CR, H & Power-Law & Without & c & 0.59 (0.54, 0.64) & - & 0.17 (0.13, 0.21)\\
\addlinespace
Gymnosperm & CR, H & Power-Law & With & a & 0.97 (0.90, 1.04) & -0.05 (-0.12, 0.03) & 0.27 (0.22, 0.34)\\
Gymnosperm & CR, H & Power-Law & With & b & 0.41 (0.37, 0.44) & 0.03 (-0.008, 0.06) & 0.11 (0.08, 0.15)\\
Gymnosperm & CR, H & Power-Law & With & c & 0.59 (0.54, 0.64) & -0.04 (-0.08, 0.01) & 0.16 (0.13, 0.21)\\
Gymnosperm & CR × H & Power-Law & Without & a & 0.99 (0.92, 1.07) & - & 0.28 (0.23, 0.35)\\
Gymnosperm & CR × H & Power-Law & Without & b & 0.89 (0.84, 0.94) & - & 0.18 (0.15, 0.23)\\
\addlinespace
Gymnosperm & CR & Power-Law & Without & a & 0.97 (0.89, 1.06) & - & 0.34 (0.28, 0.42)\\
Gymnosperm & CR & Power-Law & Without & b & 0.79 (0.71, 0.87) & - & 0.27 (0.22, 0.34)\\
Gymnosperm & H & Power-Law & Without & a & 1.00 (0.90, 1.09) & - & 0.35 (0.29, 0.43)\\
Gymnosperm & H & Power-Law & Without & b & 0.85 (0.79, 0.92) & - & 0.22 (0.18, 0.28)\\
\bottomrule
\end{longtable*}
\endgroup{}

\newpage

\begin{figure}[H]

\centering{

\pandocbounded{\includegraphics[keepaspectratio]{../figs/h_cr_dbh_log.png}}

}

\caption{\label{fig-compare}Allometric relationships between tree height
and crown radius with DBH of angiosperm (a, c) and gymnosperm (b, d)
based on three different models: power-law (orange), generalized
Michaelis-Menten (green), and Weibull (blue). The density of overlapping
points is shown using a colour gradient ranging from black (low density)
to yellow (high density). Point density was calculated using a 2D kernel
density estimation on a 300 \(\times\) 300 grid, estimating the
concentration of data points in the log-transformed space of DBH and
tree height or crown radius. All axes are on the log scale.}

\end{figure}%

\newpage

\section{Section S2. Stan code used to fit alternative allometric
models}\label{section-s2.-stan-code-used-to-fit-alternative-allometric-models-1}

\subsection{Power-law model}\label{power-law-model}

\begin{Shaded}
\begin{Highlighting}[]
\KeywordTok{data}\NormalTok{ \{}
  \DataTypeTok{int}\NormalTok{\textless{}}\KeywordTok{lower}\NormalTok{=}\DecValTok{0}\NormalTok{\textgreater{} N;                    }\CommentTok{// num trees}
  \DataTypeTok{int}\NormalTok{\textless{}}\KeywordTok{lower}\NormalTok{=}\DecValTok{1}\NormalTok{\textgreater{} K;                    }\CommentTok{// num of tree{-}level predictors}
  \DataTypeTok{int}\NormalTok{\textless{}}\KeywordTok{lower}\NormalTok{=}\DecValTok{1}\NormalTok{\textgreater{} J;                    }\CommentTok{// num sp}
  \DataTypeTok{array}\NormalTok{[N] }\DataTypeTok{int}\NormalTok{\textless{}}\KeywordTok{lower}\NormalTok{=}\DecValTok{1}\NormalTok{, }\KeywordTok{upper}\NormalTok{=J\textgreater{} jj; }\CommentTok{// sp indicator for each trees}
  \DataTypeTok{matrix}\NormalTok{[N, K] log\_x;                }\CommentTok{// tree{-}level predictors}
  \DataTypeTok{vector}\NormalTok{[N] log\_y;                   }\CommentTok{// outcomes}
\NormalTok{\}}

\KeywordTok{parameters}\NormalTok{ \{}
  \DataTypeTok{vector}\NormalTok{[K] gamma;}
  \DataTypeTok{matrix}\NormalTok{[K, J] z;}
  \DataTypeTok{vector}\NormalTok{\textless{}}\KeywordTok{lower}\NormalTok{=}\DecValTok{0}\NormalTok{\textgreater{}[K] tau;}
  \DataTypeTok{real}\NormalTok{\textless{}}\KeywordTok{lower}\NormalTok{=}\DecValTok{0}\NormalTok{\textgreater{} sigma;}
\NormalTok{\}}

\KeywordTok{transformed parameters}\NormalTok{ \{}
  \DataTypeTok{matrix}\NormalTok{[K, J] beta;}
  \ControlFlowTok{for}\NormalTok{ (k }\ControlFlowTok{in} \DecValTok{1}\NormalTok{:K) \{}
\NormalTok{    beta[k, ] = gamma[k] + tau[k] * z[k, ];}
\NormalTok{  \}}
\NormalTok{\}}

\KeywordTok{model}\NormalTok{ \{}
  \DataTypeTok{vector}\NormalTok{[N] log\_mu;}
\NormalTok{  sigma \textasciitilde{} normal(}\DecValTok{0}\NormalTok{, }\DecValTok{1}\NormalTok{);}
\NormalTok{  to\_vector(z) \textasciitilde{} std\_normal();}
\NormalTok{  tau \textasciitilde{} cauchy(}\DecValTok{0}\NormalTok{, }\FloatTok{2.5}\NormalTok{);}
\NormalTok{  gamma \textasciitilde{} normal(}\DecValTok{0}\NormalTok{, }\FloatTok{2.5}\NormalTok{);}
  \ControlFlowTok{for}\NormalTok{ (n }\ControlFlowTok{in} \DecValTok{1}\NormalTok{:N) \{}
\NormalTok{    log\_mu[n] = log\_x[n, ] * beta[, jj[n]];}
\NormalTok{  \}}
\NormalTok{  log\_y \textasciitilde{} normal(log\_mu, sigma);}
\NormalTok{\}}

\KeywordTok{generated quantities}\NormalTok{ \{}
  \DataTypeTok{vector}\NormalTok{[N] log\_lik;}
  \ControlFlowTok{for}\NormalTok{ (n }\ControlFlowTok{in} \DecValTok{1}\NormalTok{:N) \{}
\NormalTok{    log\_lik[n] = normal\_lpdf(log\_y[n] | log\_x[n, ] * beta[, jj[n]], sigma);}
\NormalTok{  \}}
\NormalTok{\}}
\end{Highlighting}
\end{Shaded}

\newpage

\subsection{Power-law model with wood
density}\label{power-law-model-with-wood-density}

\begin{Shaded}
\begin{Highlighting}[]
\KeywordTok{data}\NormalTok{ \{}
  \DataTypeTok{int}\NormalTok{\textless{}}\KeywordTok{lower}\NormalTok{=}\DecValTok{0}\NormalTok{\textgreater{} N;                    }\CommentTok{// num trees}
  \DataTypeTok{int}\NormalTok{\textless{}}\KeywordTok{lower}\NormalTok{=}\DecValTok{1}\NormalTok{\textgreater{} K;                    }\CommentTok{// num of tree{-}level predictors}
  \DataTypeTok{int}\NormalTok{\textless{}}\KeywordTok{lower}\NormalTok{=}\DecValTok{1}\NormalTok{\textgreater{} J;                    }\CommentTok{// num sp}
  \DataTypeTok{int}\NormalTok{\textless{}}\KeywordTok{lower}\NormalTok{=}\DecValTok{1}\NormalTok{\textgreater{} L;                    }\CommentTok{// num sp{-}level predictor}
  \DataTypeTok{array}\NormalTok{[N] }\DataTypeTok{int}\NormalTok{\textless{}}\KeywordTok{lower}\NormalTok{=}\DecValTok{1}\NormalTok{, }\KeywordTok{upper}\NormalTok{=J\textgreater{} jj; }\CommentTok{// sp indicator for each trees}
  \DataTypeTok{matrix}\NormalTok{[N, K] log\_x;                }\CommentTok{// tree{-}level predictors}
  \DataTypeTok{matrix}\NormalTok{[L, J] u;                    }\CommentTok{// sp{-}level predictors}
  \DataTypeTok{vector}\NormalTok{[N] log\_y;                   }\CommentTok{// outcomes}
\NormalTok{\}}

\KeywordTok{parameters}\NormalTok{ \{}
  \DataTypeTok{matrix}\NormalTok{[K, L] gamma;}
  \DataTypeTok{matrix}\NormalTok{[K, J] z;}
  \DataTypeTok{vector}\NormalTok{\textless{}}\KeywordTok{lower}\NormalTok{=}\DecValTok{0}\NormalTok{\textgreater{}[K] tau;}
  \DataTypeTok{real}\NormalTok{\textless{}}\KeywordTok{lower}\NormalTok{=}\DecValTok{0}\NormalTok{\textgreater{} sigma;}
\NormalTok{\}}

\KeywordTok{transformed parameters}\NormalTok{ \{}
  \DataTypeTok{matrix}\NormalTok{[K, J] beta;}
  \ControlFlowTok{for}\NormalTok{ (k }\ControlFlowTok{in} \DecValTok{1}\NormalTok{:K) \{}
\NormalTok{    beta[k, ] = gamma[k, ] * u + tau[k] * z[k, ];}
\NormalTok{  \}}
\NormalTok{\}}

\KeywordTok{model}\NormalTok{ \{}
  \DataTypeTok{vector}\NormalTok{[N] log\_mu;}
\NormalTok{  to\_vector(z) \textasciitilde{} std\_normal();}
\NormalTok{  to\_vector(gamma) \textasciitilde{} normal(}\DecValTok{0}\NormalTok{, }\FloatTok{2.5}\NormalTok{);}
\NormalTok{  tau \textasciitilde{} cauchy(}\DecValTok{0}\NormalTok{, }\FloatTok{2.5}\NormalTok{);}
  \ControlFlowTok{for}\NormalTok{ (n }\ControlFlowTok{in} \DecValTok{1}\NormalTok{:N) \{}
\NormalTok{    log\_mu[n] = log\_x[n, ] * beta[, jj[n]];}
\NormalTok{  \}}
\NormalTok{  log\_y \textasciitilde{} normal(log\_mu, sigma);}
\NormalTok{\}}

\KeywordTok{generated quantities}\NormalTok{ \{}
  \DataTypeTok{vector}\NormalTok{[N] log\_lik;}
  \ControlFlowTok{for}\NormalTok{ (n }\ControlFlowTok{in} \DecValTok{1}\NormalTok{:N) \{}
\NormalTok{    log\_lik[n] = normal\_lpdf(log\_y[n] | log\_x[n, ] * beta[, jj[n]], sigma);}
\NormalTok{  \}}
\NormalTok{\}}
\end{Highlighting}
\end{Shaded}

\newpage

\subsection{Generalized Michaelis-Menten
model}\label{generalized-michaelis-menten-model}

\begin{Shaded}
\begin{Highlighting}[]
\KeywordTok{data}\NormalTok{ \{}
  \DataTypeTok{int}\NormalTok{\textless{}}\KeywordTok{lower}\NormalTok{=}\DecValTok{0}\NormalTok{\textgreater{} N;                    }\CommentTok{// num trees}
  \DataTypeTok{int}\NormalTok{\textless{}}\KeywordTok{lower}\NormalTok{=}\DecValTok{1}\NormalTok{\textgreater{} J;                    }\CommentTok{// num sp}
  \DataTypeTok{int}\NormalTok{\textless{}}\KeywordTok{lower}\NormalTok{=}\DecValTok{1}\NormalTok{\textgreater{} K;                    }\CommentTok{// num of tree{-}level predictors}
  \DataTypeTok{array}\NormalTok{[N] }\DataTypeTok{int}\NormalTok{\textless{}}\KeywordTok{lower}\NormalTok{=}\DecValTok{1}\NormalTok{, }\KeywordTok{upper}\NormalTok{=J\textgreater{} jj; }\CommentTok{// sp indicator for each tree}
  \DataTypeTok{vector}\NormalTok{[N] x;                       }\CommentTok{// DBH (non{-}log{-}scale)}
  \DataTypeTok{vector}\NormalTok{[N] log\_y;                   }\CommentTok{// outcomes}
\NormalTok{\}}

\KeywordTok{parameters}\NormalTok{ \{}
  \DataTypeTok{vector}\NormalTok{\textless{}}\KeywordTok{lower}\NormalTok{=}\DecValTok{0}\NormalTok{\textgreater{}[K] gamma\_hat;}
  \DataTypeTok{matrix}\NormalTok{[J, K] z;}
  \DataTypeTok{vector}\NormalTok{\textless{}}\KeywordTok{lower}\NormalTok{=}\DecValTok{0}\NormalTok{\textgreater{}[K] tau;}
  \DataTypeTok{real}\NormalTok{\textless{}}\KeywordTok{lower}\NormalTok{=}\DecValTok{0}\NormalTok{\textgreater{} sigma;}
\NormalTok{\}}

\KeywordTok{transformed parameters}\NormalTok{ \{}
  \DataTypeTok{matrix}\NormalTok{[J, K] beta;}
  \DataTypeTok{vector}\NormalTok{[K] gamma;}
\NormalTok{  gamma[}\DecValTok{1}\NormalTok{] = gamma\_hat[}\DecValTok{1}\NormalTok{] * }\DecValTok{10}\NormalTok{;}
\NormalTok{  gamma[}\DecValTok{2}\NormalTok{] = gamma\_hat[}\DecValTok{2}\NormalTok{];}
\NormalTok{  gamma[}\DecValTok{3}\NormalTok{] = gamma\_hat[}\DecValTok{3}\NormalTok{] * }\DecValTok{1000}\NormalTok{;}
  \CommentTok{// Corrected matrix multiplication and addition}
  \ControlFlowTok{for}\NormalTok{ (k }\ControlFlowTok{in} \DecValTok{1}\NormalTok{:K) \{}
\NormalTok{    beta[, k] =  gamma[k] + tau[k] * z[, k];}
\NormalTok{  \}}
\NormalTok{\}}

\KeywordTok{model}\NormalTok{ \{}
  \DataTypeTok{vector}\NormalTok{[N] log\_mu;}
\NormalTok{  sigma \textasciitilde{} normal(}\DecValTok{0}\NormalTok{, }\DecValTok{1}\NormalTok{);}
\NormalTok{  to\_vector(z) \textasciitilde{} std\_normal();}
\NormalTok{  tau \textasciitilde{} cauchy(}\DecValTok{0}\NormalTok{, }\FloatTok{2.5}\NormalTok{);}
\NormalTok{  gamma\_hat \textasciitilde{} normal(}\DecValTok{0}\NormalTok{, }\FloatTok{2.5}\NormalTok{);}
\NormalTok{  log\_mu = beta[jj, }\DecValTok{1}\NormalTok{] + beta[jj, }\DecValTok{2}\NormalTok{] .* log(x) {-}}
\NormalTok{      log(beta[jj, }\DecValTok{3}\NormalTok{] + pow(x, beta[jj, }\DecValTok{2}\NormalTok{]));}
\NormalTok{  log\_y \textasciitilde{} normal(log\_mu, sigma);}
\NormalTok{\}}

\KeywordTok{generated quantities}\NormalTok{ \{}
  \DataTypeTok{vector}\NormalTok{[N] log\_lik;}
  \ControlFlowTok{for}\NormalTok{ (n }\ControlFlowTok{in} \DecValTok{1}\NormalTok{:N) \{}
\NormalTok{    log\_lik[n] = normal\_lpdf(log\_y[n] |}
\NormalTok{      beta[jj[n], }\DecValTok{1}\NormalTok{] +}
\NormalTok{      beta[jj[n], }\DecValTok{2}\NormalTok{] * log(x[n]) {-}}
\NormalTok{      log(beta[jj[n], }\DecValTok{3}\NormalTok{] + pow(x[n], beta[jj[n], }\DecValTok{2}\NormalTok{])),}
\NormalTok{      sigma);}
\NormalTok{  \}}
\NormalTok{\}}
\end{Highlighting}
\end{Shaded}

\newpage

\subsection{Generalized Michaelis-Menten model with wood
density}\label{generalized-michaelis-menten-model-with-wood-density}

\begin{Shaded}
\begin{Highlighting}[]
\KeywordTok{data}\NormalTok{ \{}
  \DataTypeTok{int}\NormalTok{\textless{}}\KeywordTok{lower}\NormalTok{=}\DecValTok{0}\NormalTok{\textgreater{} N;                    }\CommentTok{// num trees}
  \DataTypeTok{int}\NormalTok{\textless{}}\KeywordTok{lower}\NormalTok{=}\DecValTok{1}\NormalTok{\textgreater{} J;                    }\CommentTok{// num sp}
  \DataTypeTok{int}\NormalTok{\textless{}}\KeywordTok{lower}\NormalTok{=}\DecValTok{1}\NormalTok{\textgreater{} K;                    }\CommentTok{// num of tree{-}level predictors}
  \DataTypeTok{int}\NormalTok{\textless{}}\KeywordTok{lower}\NormalTok{=}\DecValTok{1}\NormalTok{\textgreater{} L;                    }\CommentTok{// num sp{-}level predictor}
  \DataTypeTok{array}\NormalTok{[N] }\DataTypeTok{int}\NormalTok{\textless{}}\KeywordTok{lower}\NormalTok{=}\DecValTok{1}\NormalTok{, }\KeywordTok{upper}\NormalTok{=J\textgreater{} jj; }\CommentTok{// sp indicator for each tree}
  \DataTypeTok{vector}\NormalTok{[N] x;                       }\CommentTok{// DBH (non{-}log{-}scale)}
  \DataTypeTok{matrix}\NormalTok{[J, L] u;                    }\CommentTok{// sp{-}level predictors}
  \DataTypeTok{vector}\NormalTok{[N] log\_y;                   }\CommentTok{// outcomes}
\NormalTok{\}}

\KeywordTok{parameters}\NormalTok{ \{}
  \DataTypeTok{vector}\NormalTok{\textless{}}\KeywordTok{lower}\NormalTok{=}\DecValTok{0}\NormalTok{\textgreater{}[K] gamma\_int\_hat;}
  \DataTypeTok{vector}\NormalTok{[K] gamma\_slope\_hat;}
  \DataTypeTok{matrix}\NormalTok{[J, K] z;}
  \DataTypeTok{vector}\NormalTok{\textless{}}\KeywordTok{lower}\NormalTok{=}\DecValTok{0}\NormalTok{\textgreater{}[K] tau;}
  \DataTypeTok{real}\NormalTok{\textless{}}\KeywordTok{lower}\NormalTok{=}\DecValTok{0}\NormalTok{\textgreater{} sigma;}
\NormalTok{\}}

\KeywordTok{transformed parameters}\NormalTok{ \{}
  \DataTypeTok{matrix}\NormalTok{[J, K] beta;}
  \DataTypeTok{row\_vector}\NormalTok{[K] gamma\_int;}
  \DataTypeTok{row\_vector}\NormalTok{[K] gamma\_slope;}
\NormalTok{  gamma\_int[}\DecValTok{1}\NormalTok{] = gamma\_int\_hat[}\DecValTok{1}\NormalTok{] * }\DecValTok{10}\NormalTok{;}
\NormalTok{  gamma\_int[}\DecValTok{2}\NormalTok{] = gamma\_int\_hat[}\DecValTok{2}\NormalTok{];}
\NormalTok{  gamma\_int[}\DecValTok{3}\NormalTok{] = gamma\_int\_hat[}\DecValTok{3}\NormalTok{] * }\DecValTok{1000}\NormalTok{;}
\NormalTok{  gamma\_slope[}\DecValTok{1}\NormalTok{] = gamma\_slope\_hat[}\DecValTok{1}\NormalTok{] * }\DecValTok{10}\NormalTok{;}
\NormalTok{  gamma\_slope[}\DecValTok{2}\NormalTok{] = gamma\_slope\_hat[}\DecValTok{2}\NormalTok{];}
\NormalTok{  gamma\_slope[}\DecValTok{3}\NormalTok{] = gamma\_slope\_hat[}\DecValTok{3}\NormalTok{] * }\DecValTok{1000}\NormalTok{;}
  \DataTypeTok{matrix}\NormalTok{[}\DecValTok{2}\NormalTok{, K] gamma = append\_row(gamma\_int, gamma\_slope);}

  \CommentTok{// Corrected matrix multiplication and addition}
  \ControlFlowTok{for}\NormalTok{ (k }\ControlFlowTok{in} \DecValTok{1}\NormalTok{:K) \{}
\NormalTok{    beta[, k] = u * gamma[, k] + tau[k] * z[, k];}
\NormalTok{  \}}
\NormalTok{\}}

\KeywordTok{model}\NormalTok{ \{}
  \DataTypeTok{vector}\NormalTok{[N] log\_mu;}
\NormalTok{  to\_vector(z) \textasciitilde{} std\_normal();}
\NormalTok{  tau \textasciitilde{} cauchy(}\DecValTok{0}\NormalTok{, }\FloatTok{2.5}\NormalTok{);}
\NormalTok{  gamma\_int\_hat \textasciitilde{} normal(}\DecValTok{0}\NormalTok{, }\FloatTok{2.5}\NormalTok{);}
\NormalTok{  gamma\_slope\_hat \textasciitilde{} normal(}\DecValTok{0}\NormalTok{, }\FloatTok{1.25}\NormalTok{);}
  \ControlFlowTok{for}\NormalTok{ (n }\ControlFlowTok{in} \DecValTok{1}\NormalTok{:N) \{}
\NormalTok{    log\_mu[n] = beta[jj[n], }\DecValTok{1}\NormalTok{] + beta[jj[n], }\DecValTok{2}\NormalTok{] * log(x[n]) {-}}
\NormalTok{      log(beta[jj[n], }\DecValTok{3}\NormalTok{] + pow(x[n], beta[jj[n], }\DecValTok{2}\NormalTok{]));}
\NormalTok{  \}}
\NormalTok{  log\_y \textasciitilde{} normal(log\_mu, sigma);}
\NormalTok{\}}

\KeywordTok{generated quantities}\NormalTok{ \{}
  \DataTypeTok{vector}\NormalTok{[N] log\_lik;}
  \ControlFlowTok{for}\NormalTok{ (n }\ControlFlowTok{in} \DecValTok{1}\NormalTok{:N) \{}
\NormalTok{    log\_lik[n] = normal\_lpdf(log\_y[n] |}
\NormalTok{      beta[jj[n], }\DecValTok{1}\NormalTok{] +}
\NormalTok{      beta[jj[n], }\DecValTok{2}\NormalTok{] * log(x[n]) {-}}
\NormalTok{      log(beta[jj[n], }\DecValTok{3}\NormalTok{] + pow(x[n], beta[jj[n], }\DecValTok{2}\NormalTok{])),}
\NormalTok{      sigma);}
\NormalTok{  \}}
\NormalTok{\}}
\end{Highlighting}
\end{Shaded}

\newpage

\subsection{Weibull model}\label{weibull-model}

\begin{Shaded}
\begin{Highlighting}[]
\KeywordTok{data}\NormalTok{ \{}
  \DataTypeTok{int}\NormalTok{\textless{}}\KeywordTok{lower}\NormalTok{=}\DecValTok{0}\NormalTok{\textgreater{} N;                    }\CommentTok{// num trees}
  \DataTypeTok{int}\NormalTok{\textless{}}\KeywordTok{lower}\NormalTok{=}\DecValTok{1}\NormalTok{\textgreater{} J;                    }\CommentTok{// num sp}
  \DataTypeTok{int}\NormalTok{\textless{}}\KeywordTok{lower}\NormalTok{=}\DecValTok{1}\NormalTok{\textgreater{} K;                    }\CommentTok{// num of tree{-}level predictors}
  \DataTypeTok{array}\NormalTok{[N] }\DataTypeTok{int}\NormalTok{\textless{}}\KeywordTok{lower}\NormalTok{=}\DecValTok{1}\NormalTok{, }\KeywordTok{upper}\NormalTok{=J\textgreater{} jj; }\CommentTok{// sp indicator for each tree}
  \DataTypeTok{vector}\NormalTok{[N] x;                       }\CommentTok{// DBH (non{-}log{-}scale)}
  \DataTypeTok{vector}\NormalTok{[N] log\_y;                   }\CommentTok{// outcomes}
\NormalTok{\}}

\KeywordTok{parameters}\NormalTok{ \{}
  \DataTypeTok{vector}\NormalTok{\textless{}}\KeywordTok{lower}\NormalTok{=}\DecValTok{0}\NormalTok{\textgreater{}[K] gamma;}
  \DataTypeTok{matrix}\NormalTok{[J, K] z;}
  \DataTypeTok{vector}\NormalTok{\textless{}}\KeywordTok{lower}\NormalTok{=}\DecValTok{0}\NormalTok{\textgreater{}[K] tau;}
  \DataTypeTok{real}\NormalTok{\textless{}}\KeywordTok{lower}\NormalTok{=}\DecValTok{0}\NormalTok{\textgreater{} sigma;}
\NormalTok{\}}

\KeywordTok{transformed parameters}\NormalTok{ \{}
  \DataTypeTok{matrix}\NormalTok{[J, K] beta;}
  \CommentTok{// Corrected matrix multiplication and addition}
  \ControlFlowTok{for}\NormalTok{ (k }\ControlFlowTok{in} \DecValTok{1}\NormalTok{:K) \{}
\NormalTok{    beta[, k] =  gamma[k] + tau[k] * z[, k];}
\NormalTok{  \}}
\NormalTok{\}}

\KeywordTok{model}\NormalTok{ \{}
  \DataTypeTok{vector}\NormalTok{[N] log\_mu;}
\NormalTok{  sigma \textasciitilde{} normal(}\DecValTok{0}\NormalTok{, }\DecValTok{1}\NormalTok{);}
\NormalTok{  to\_vector(z) \textasciitilde{} std\_normal();}
\NormalTok{  tau \textasciitilde{} cauchy(}\DecValTok{0}\NormalTok{, }\FloatTok{2.5}\NormalTok{);}
\NormalTok{  gamma \textasciitilde{} normal(}\DecValTok{0}\NormalTok{, }\FloatTok{2.5}\NormalTok{);}
\NormalTok{  log\_mu = beta[jj, }\DecValTok{1}\NormalTok{] + log1m\_exp({-}beta[jj, }\DecValTok{2}\NormalTok{] .* pow(x, beta[jj, }\DecValTok{3}\NormalTok{]));}
\NormalTok{  log\_y \textasciitilde{} normal(log\_mu, sigma);}
\NormalTok{\}}

\KeywordTok{generated quantities}\NormalTok{ \{}
  \DataTypeTok{vector}\NormalTok{[N] log\_lik;}
  \ControlFlowTok{for}\NormalTok{ (n }\ControlFlowTok{in} \DecValTok{1}\NormalTok{:N) \{}
\NormalTok{    log\_lik[n] = normal\_lpdf(log\_y[n] |}
\NormalTok{      beta[jj[n], }\DecValTok{1}\NormalTok{] + log1m\_exp({-}beta[jj[n], }\DecValTok{2}\NormalTok{] * pow(x[n], beta[jj[n], }\DecValTok{3}\NormalTok{])),}
\NormalTok{      sigma);}
\NormalTok{  \}}
\NormalTok{\}}
\end{Highlighting}
\end{Shaded}

\newpage

\subsection{Weibull model with wood
density}\label{weibull-model-with-wood-density}

\begin{Shaded}
\begin{Highlighting}[]
\KeywordTok{data}\NormalTok{ \{}
  \DataTypeTok{int}\NormalTok{\textless{}}\KeywordTok{lower}\NormalTok{=}\DecValTok{0}\NormalTok{\textgreater{} N;                    }\CommentTok{// num trees}
  \DataTypeTok{int}\NormalTok{\textless{}}\KeywordTok{lower}\NormalTok{=}\DecValTok{1}\NormalTok{\textgreater{} J;                    }\CommentTok{// num sp}
  \DataTypeTok{int}\NormalTok{\textless{}}\KeywordTok{lower}\NormalTok{=}\DecValTok{1}\NormalTok{\textgreater{} K;                    }\CommentTok{// num of tree{-}level predictors}
  \DataTypeTok{int}\NormalTok{\textless{}}\KeywordTok{lower}\NormalTok{=}\DecValTok{1}\NormalTok{\textgreater{} L;                    }\CommentTok{// num sp{-}level predictor}
  \DataTypeTok{array}\NormalTok{[N] }\DataTypeTok{int}\NormalTok{\textless{}}\KeywordTok{lower}\NormalTok{=}\DecValTok{1}\NormalTok{, }\KeywordTok{upper}\NormalTok{=J\textgreater{} jj; }\CommentTok{// sp indicator for each tree}
  \DataTypeTok{vector}\NormalTok{[N] x;                       }\CommentTok{// DBH (non{-}log{-}scale)}
  \DataTypeTok{matrix}\NormalTok{[J, L] u;                    }\CommentTok{// sp{-}level predictors}
  \DataTypeTok{vector}\NormalTok{[N] log\_y;                   }\CommentTok{// outcomes}
\NormalTok{\}}

\KeywordTok{parameters}\NormalTok{ \{}
  \DataTypeTok{row\_vector}\NormalTok{\textless{}}\KeywordTok{lower}\NormalTok{=}\DecValTok{0}\NormalTok{\textgreater{}[K] gamma\_int;}
  \DataTypeTok{row\_vector}\NormalTok{[K] gamma\_slope;}
  \DataTypeTok{matrix}\NormalTok{[J, K] z;}
  \DataTypeTok{vector}\NormalTok{\textless{}}\KeywordTok{lower}\NormalTok{=}\DecValTok{0}\NormalTok{\textgreater{}[K] tau;}
  \DataTypeTok{real}\NormalTok{\textless{}}\KeywordTok{lower}\NormalTok{=}\DecValTok{0}\NormalTok{\textgreater{} sigma;}
\NormalTok{\}}

\KeywordTok{transformed parameters}\NormalTok{ \{}
  \DataTypeTok{matrix}\NormalTok{[J, K] beta;}
  \DataTypeTok{matrix}\NormalTok{[}\DecValTok{2}\NormalTok{, K] gamma = append\_row(gamma\_int, gamma\_slope);}

  \CommentTok{// Corrected matrix multiplication and addition}
  \ControlFlowTok{for}\NormalTok{ (k }\ControlFlowTok{in} \DecValTok{1}\NormalTok{:K) \{}
\NormalTok{    beta[, k] = u * gamma[, k] + tau[k] * z[, k];}
\NormalTok{  \}}
\NormalTok{\}}

\KeywordTok{model}\NormalTok{ \{}
  \DataTypeTok{vector}\NormalTok{[N] log\_mu;}
\NormalTok{  to\_vector(z) \textasciitilde{} std\_normal();}
\NormalTok{  to\_vector(gamma\_int) \textasciitilde{} normal(}\DecValTok{0}\NormalTok{, }\FloatTok{2.5}\NormalTok{);}
\NormalTok{  to\_vector(gamma\_slope) \textasciitilde{} normal(}\DecValTok{0}\NormalTok{, }\FloatTok{1.25}\NormalTok{);}
\NormalTok{  to\_vector(tau) \textasciitilde{} cauchy(}\DecValTok{0}\NormalTok{, }\FloatTok{2.5}\NormalTok{);}
  \ControlFlowTok{for}\NormalTok{ (n }\ControlFlowTok{in} \DecValTok{1}\NormalTok{:N) \{}
\NormalTok{    log\_mu[n] = beta[jj[n], }\DecValTok{1}\NormalTok{] + log1m\_exp({-}beta[jj[n], }\DecValTok{2}\NormalTok{] * pow(x[n], beta[jj[n], }\DecValTok{3}\NormalTok{]));}
\NormalTok{  \}}
\NormalTok{  log\_y \textasciitilde{} normal(log\_mu, sigma);}
\NormalTok{\}}

\KeywordTok{generated quantities}\NormalTok{ \{}
  \DataTypeTok{vector}\NormalTok{[N] log\_lik;}
  \ControlFlowTok{for}\NormalTok{ (n }\ControlFlowTok{in} \DecValTok{1}\NormalTok{:N) \{}
\NormalTok{    log\_lik[n] = normal\_lpdf(log\_y[n] |}
\NormalTok{      beta[jj[n], }\DecValTok{1}\NormalTok{] + log1m\_exp({-}beta[jj[n], }\DecValTok{2}\NormalTok{] * pow(x[n], beta[jj[n], }\DecValTok{3}\NormalTok{])),}
\NormalTok{      sigma);}
\NormalTok{  \}}
\NormalTok{\}}
\end{Highlighting}
\end{Shaded}





\end{document}
